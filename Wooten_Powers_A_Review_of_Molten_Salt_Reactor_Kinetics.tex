\documentclass[review]{elsarticle}

\usepackage{lineno,hyperref}
\modulolinenumbers[5]

%For inclusion of figures
\usepackage{graphicx}

%For isotope notation
\usepackage{mhchem}

%For equation formatting
\usepackage{amsmath}

%For sizing summations properly
\usepackage{relsize}

%For tables
\usepackage{multirow}

%For signalling my co-author
\usepackage{xcolor}

%For landscape tables
\usepackage{pdflscape}

%For footnotes in tables
\usepackage{tablefootnote}

%For appendicies
\usepackage[page]{appendix}

%For fixing floats in place
\usepackage{float}

\journal{Progress in Nuclear Energy}

%%%%%%%%%%%%%%%%%%%%%%%
%% Elsevier bibliography styles
%%%%%%%%%%%%%%%%%%%%%%%
%% To change the style, put a % in front of the second line of the current style and
%% remove the % from the second line of the style you would like to use.
%%%%%%%%%%%%%%%%%%%%%%%

%% Numbered
%\bibliographystyle{model1-num-names}

%% Numbered without titles
%\bibliographystyle{model1a-num-names}

%% Harvard
%\bibliographystyle{model2-names.bst}\biboptions{authoryear}

%% Vancouver numbered
%\usepackage{numcompress}\bibliographystyle{model3-num-names}

%% Vancouver name/year
%\usepackage{numcompress}\bibliographystyle{model4-names}\biboptions{authoryear}

%% APA style
%\bibliographystyle{model5-names}\biboptions{authoryear}

%% AMA style
%\usepackage{numcompress}\bibliographystyle{model6-num-names}

%% `Elsevier LaTeX' style
\bibliographystyle{elsarticle-num}
%%%%%%%%%%%%%%%%%%%%%%%

\begin{document}

\begin{frontmatter}

\title{A Review of Molten Salt Reactor Kinetics Models}

%% Group authors per affiliation:
\author[ucb]{Daniel Wooten\corref{cor1}}
\ead{danieldavidwooten@gmail.com}
\address[ucb]{4155 Etcheverry Hall, MC 1730, University of California, Berkeley,
    Berkeley, CA 94720-1730}
\cortext[cor1]{Corresponding Author}

\author[ornl]{Jeffrey Powers}
\address[ornl]{Oak Ridge}

\begin{abstract}
Interest in circulating fuel reactors, particularly molten salt reactors (MSR)
 of the
fluid fuel type, has been growing in the last two decades starting with a
resurgence of interest in Europe. Since this time there have been a growing
number of methods proposed and codes developed to model the kinetics of CFRs;
a capability essential to the design and evaluation of such reactors. This work
first reviews the physical phenomenon unique to CFRs in the light of current
research and how the kinetics of CFRs are impacted by these considerations.
 An extensive review of published models and methods for simulating CFR kinetics
is presented along with transient simulations in fast and thermal systems using
representative codes from each of the main modelling categories. Finally a
review of common assumptions used in these models is presented along with an
evaluation of their impact on model performance. 
\end{abstract}

\begin{keyword}
\texttt{circulating fluid kinetics precursor}
\MSC[2010] 00-01\sep  99-00
\end{keyword}

\end{frontmatter}

\linenumbers

\section{Introduction} \label{sec:intro}
\textcolor{red}{JEFF, COULD YOU INCLUDE SOMETHING ABOUT THE AQUEOUS REACTOR IN HERE? Also, I'm not particularly gifted with intros, so please do have at this}
Circulating fuel reactors (CFRs) have a long and somewhat quiet history.
This history began with the Aircraft Reactor Experiment at Oak Ridge National
Laboratories in the late 50s and grew with the Molten Salt Reactor Experiment
(MSRE) which successfully built and operated an 8 MW molten salt reactor (MSR) for 
more than 13,000 equivalent full-power hours with both \ce{^{233}U} and \ce{^{235}U} fuel at different
points in time. With the closing of the MSRE project in the late 60s the world
saw very little interest in and development of CFRs or MSRs. This changed in
the late 90s as interest began growing throughout Europe. This momentum
grew with the MOST project and continued through ALISIA and EVOL resulting in
the pre-conceptual Molten Salt Fast Reactor (MSFR). The Molten Salt Actinide
Recycler and Transmuter (MOSART), a design largely out of Russia, the
Liquid Fueled Thorium Molten Salt Reactor (LFTMSR), a Chinese design, the
FUJI series of reactors out of Japan, and a host of proposed privately designed
reactors have all additionally come into the world view and driven demand for
the simulation of CFRs. 
\par The simulation of CFRs proves to not be a trivial task even with today's
large spread of reactor simulation tools and models. This difficulty rests on
the many differences between solid fuel reactors, which have received the
majority of global attention, and CFRs, namely; a liquid fuel which flows in
the core, natural processes which remove specific elements from the liquid fuel,
and the ability to perform online reprocessing of the liquid fuel. It is this
first unique aspect, a flowing fuel, which this work, and the works it looks at,
are primarily concerned with. Kinetics modelling, or the ability to predict
a reactor's response in time scales typically less than a few hours, is a key
requirement to any reactor design effort. This modelling is challenged in CFRs
specifically because of their flowing fuel which not only can act as a coolant
for any moderator a design might have, but which displaces the delayed
neutron precursors (DNPs) from their originating positions. In a solid fuel
reactor DNPs are typically treated as simply a time-lagged source of neutrons.
In CFRs DNPs are displaced from their originating position, most often locations
of high neutronic importance, and moved throughout the reactor's primary loop
where they may decay in any number of places with varying amounts of
neutronic importance, perhaps even zero neutronic importance. 
\par It is this movement of the DNPs which CFR kinetics models generally
attempt to account for. Approaches for doing so range from solving the 
time resolved multi-group neutron diffusion equations with advective and
diffusive terms included on the DNP balance equations all the way to adding
source and sink terms to the point reactor kinetics equations to account
for out of core decay of DNPs. Considering the tight interplay between the
fluid motion and the neutronic behavior in a CFR many CFR kinetics approaches
include coupling, loose and tight, to thermohydraulics simulations. In the
following sections a review and discussion, guided by recent research, of the
unique physical phenomenon tied to CFR kinetics modeling will be given. 
Following an extended discussion of the various and proposed models for CFR
kinetics analysis will be given along with sample transient simulations for
both fast and thermal systems using representative codes from each of the main
simulation method groups. Lastly, common assumptions and their affects on CFR
kinetics modelling will be investigated.
\par A list of abbreviations used in this paper can be found in table
\ref{tbl:nomen} while a summarization of all computational tools analyzed
in this work can be found in tables \ref{tbl:codes1} through \ref{tbl:codes5}
 with a description of said
tables in \ref{app:tools}.

\begin{table}[h]
    \caption{A listing of key acronyms used in this paper}.
    \label{tbl:nomen}
    \begin{center}
        \begin{tabular}{|c c|}
            \hline
            CPK & Corrected Point Reactor Kinetics \\
            CVM & Control Volume Method \\
            DNP & Delayed Neutron Precursor \\
            DPK & Delayed Point Reactor Kinetics \\
            DVM & Discrete Volume Method \\
            ESDIRK & Explicit Singly Diagonally Implicit Runge-Kutta \\
            FDM & Finite Differences Method \\
            FED & Finite Element Discritization \\
            FEM & Finite Element Method \\
            FID & Fully Implicit Discritization \\
            FVM & Finite Volume Method \\
            IDE & Implicit Discrete Euler \\
            IEM & Implicit Euler Method \\
            IFD & Implicit Finite Difference \\
            IFP & Iterated Fission Probability \\
            IPK & ``I" Point Reactor Kinetics \\
            MGD & Multi-Group Neutron Diffusion \\
            MMC & Modified Monte Carlo ( accounting for DNP drift ) \\
            MOSART & Molten Salt Actinide Recycler and Transumter \\
            MPK & Modified Point Reactor Kinetics \\
            MSBR & Molten Salt Breeder Reactor \\
            MSFR & Molten Salt Fast Reactor \\
            MSRE & Molten Salt Reactor Experiment \\
            N & Neutronics \\
            NEM & Nodal Expansion Method \\
            OTFDB & On The Fly Doppler Broadening \\
            PRK & Point Reactor Kinetics \\
            RANS & Reynolds-Averaged Navier-Stokes \\
            SPK & Separated Point Reactor Kinetics \\
            TH & Thermohydraulics \\
            TMSR-LF & Thorium Molten Salt Reactor - Liquid Fuel \\
            Uk. & Unknown \\
            VNM & Variational Nodal Method \\
            \hline
        \end{tabular}
    \end{center}
\end{table}

\section{Physics of Circulating Fuel} \label{sec:physics}
In comparison to solid fuel reactors circulating fuel reactors exhibit unique
physics which affect all aspects of reactor kinetics. In the following
sub-sections the key phenomenon unique to the kinetics of circulating fuel
reactors are elaborated upon.

\subsection{Neutron flux} \label{ssec:flux}
While the speed of neutrons, fast or thermal, is typically many orders of magnitude greater than
the flow velocity of the fluid fuel the neutron flux can still experience
perturbations due to this fluid velocity field. These perturbations are always
due to second order effects; generally related to changes in fluid temperature
and distributions of delayed neutron precursors - an affect which will be
discussed in the next sub-section. In \cite{zhang_development_2009-1} a steady
state neutronics and coupled thermohydraulics code, which includes convective
fluid motion in the neutron flux, is used to investigate the effects of fluid
flow on various quantities of interest related to circulating fuel reactors. In
figure \ref{fig:zhang_2d_flux} \cite{zhang_development_2009-1} shows that the
distribution of the neutron flux, thermal is shown but the conclusions for the fast flux are the same, is not significantly shifted in the direction
of the fluid flow. However, \cite{zhang_development_2009-1} does show in figure \ref{fig:zhang_axial_velocity_flux}, axially, and in figure \ref{fig:zhang_radial_velocity_flux}, radially, how a greater fluid flow
 velocity, $U_{in}$ in the figures, suppresses the magnitude of the fast, $\phi_{1}$, and thermal, $\phi_{2}$, neutron fluxes. This affect is
 attributed to a larger transport of delayed neutron precursors outside of the core \cite{zhang_development_2009-1}. The reactor model
 utilized in \cite{zhang_development_2009-1} consists of an axially symmetric
 open core type reactor with an outer graphite reflector. In the analyses which
 produced figures \ref{fig:zhang_axial_velocity_flux} and \ref{fig:zhang_radial_velocity_flux} the
 out-of-core residence time for the fuel was held constant despite a changing
 fluid flow velocity.

\begin{figure}[h]
   \centering
   \includegraphics[width=0.8\textwidth]{Zhang_Development_2009-1_Fig_8_B}
   \caption{Figure 8.b from \cite{zhang_development_2009-1}. Contour line
    values are the neutron flux; no units were provided with these values.} 
   \label{fig:zhang_2d_flux}
\end{figure}

\begin{figure}[h]
   \centering
   \includegraphics[width=0.8\textwidth]{Zhang_Development_2009-1_Fig_15_A}
   \caption{Figure 15.a from \cite{zhang_development_2009-1}. 
               Axial
               values are taken at the center of the core.}
   \label{fig:zhang_axial_velocity_flux}
\end{figure}

\begin{figure}[h]
   \centering
   \includegraphics[width=0.8\textwidth]{Zhang_Development_2009-1_Fig_15_B}
   \caption{Figure 15.b from \cite{zhang_development_2009-1}.
               Radial 
               values are taken at the center of the core.}
   \label{fig:zhang_radial_velocity_flux}
\end{figure}

\subsection{Delayed neutron precursor distribution} \label{ssec:dnpd}
In contrast to solid fuel reactors where delayed neutron precursors (DNP) decay
 very close to the positions in which they were born, DNPs in CFRs are displaced
 from their originating positions by fluid flow effects; whether advective or
 diffusive. Furthermore, considering that the transit time through the entire
 core and flow loop system of many proposed CFRs is comparable to the decay
 constants of, 
 at least, the longer DNP groups a significant fraction of these DNPs can be
expected to decay outside of the core region; emitting their neutrons in areas
of low or zero neutronic importance. As such $\beta_{eff}$, the reactivity
 introduced into a multiplying system by the neutrons emitted by DNPs
is reduced. In the \ce{^{235}U} fueled MSRE the reduction in $\beta_{eff}$ at
zero power was 212 pcm \cite{delpech_benchmark_2003}.

\par Not only do fluid flow effects displace DNPs from their originating positions,
as shown in figure \ref{fig:dulla_msre_dnp_displacement}, these effects also
reduce the neutronic importance of DNPs as shown in figure
\ref{fig:dulla_msre_dnp_importance}. This importance reduction is due to two
effects; the likelihood to decay out of core and the likelihood that, even if the
DNP decays in-core, the emitted neutron will originate in a region of low
importance. In \cite{dulla_models_2005} a multi-group
diffusion scheme modeling the MSRE was used to produce the results in figures
\ref{fig:dulla_msre_dnp_displacement} and \ref{fig:dulla_msre_dnp_importance}.
The effects on DNP distributions, from \cite{zhang_development_2009-1}, caused
by varying flow velocity while holding out-of-core residence time constant are
shown, axially, in figures \ref{fig:zhang_axial_velocity_dnp_1} and 
\ref{fig:zhang_axial_velocity_dnp_2} and, radially, in figures
\ref{fig:zhang_radial_velocity_dnp_1} and \ref{fig:zhang_radial_velocity_dnp_2}.
It is seen that the DNP groups with longer half-lives are most affected as the
fluid flow has sufficient time to displace them before they decay. For all
affected groups it is seen that a higher fluid velocity reduces DNP group
concentration in the core; more DNPs being moved out of the core where they
eventually decay. In figure \ref{fig:yamamoto_transit_time}, while holding fluid flow velocity constant, the effect of
out-of-core residence time on DNP group concentration
in core is shown. It is seen that with increasing transit time the concentration
of DNP groups in core decreases as more DNPs decay out of core.   

\begin{figure}[h]
   \centering
   \includegraphics[width=0.8\textwidth]{Dulla_Models_2005_Fig_1_9}
   \caption{Figure 1.9 from \cite{dulla_models_2005}. Dashed lines represent
    solid fuel results while solid lines represent flowing fuel results. Values
    are taken in the middle of the core. $C_{n}$ are DNP group concentrations. No
    units were given. Higher n indicates shorter decay half-life.}
   \label{fig:dulla_msre_dnp_displacement}
\end{figure}

\begin{figure}[h]
   \centering
   \includegraphics[width=0.8\textwidth]{Dulla_Models_2005_Fig_2_4}
   \caption{Figure 2.4 from \cite{dulla_models_2005}. Dashed lines represent
    solid fuel results while solid lines represent flowing fuel results. Values
    are taken in the middle of the core. $C_{n}^{+}$ are DNP group importance values. 
    No units were given. Higher n indicates shorter decay half-life.}
   \label{fig:dulla_msre_dnp_importance}
\end{figure}

\begin{figure}[h]
   \centering
   \includegraphics[width=0.8\textwidth]{Zhang_Development_2009-1_Fig_16_A}
   \caption{Figure 16.a from \cite{zhang_development_2009-1}. 
               Axial
               values are taken at the center of the core. $C_{n}$ are DNP
               group concentrations. No units were given. Higher n indicates
               shorter decay half-life.}
   \label{fig:zhang_axial_velocity_dnp_1}
\end{figure}

\begin{figure}[h]
   \centering
   \includegraphics[width=0.8\textwidth]{Zhang_Development_2009-1_Fig_16_C}
   \caption{Figure 16.c from \cite{zhang_development_2009-1}. 
               Axial
               values are taken at the center of the core. $C_{n}$ are DNP
               group concentrations. No units were given. Higher n indicates
               shorter decay half-life.}
   \label{fig:zhang_axial_velocity_dnp_2}
\end{figure}

\begin{figure}[h]
   \centering
   \includegraphics[width=0.8\textwidth]{Zhang_Development_2009-1_Fig_16_B}
   \caption{Figure 16.b from \cite{zhang_development_2009-1}. 
               Radial 
               values are taken at the center of the core. $C_{n}$ are DNP
               group concentrations. No units were given. Higher n indicates
               shorter decay half-life.}
   \label{fig:zhang_radial_velocity_dnp_1}
\end{figure}

\begin{figure}[h]
   \centering
   \includegraphics[width=0.8\textwidth]{Zhang_Development_2009-1_Fig_16_D}
   \caption{Figure 16.d from \cite{zhang_development_2009-1}. 
               Radial
               values are taken at the center of the core. $C_{n}$ are DNP
               group concentrations. No units were given. Higher n indicates
               shorter decay half-life.}
   \label{fig:zhang_radial_velocity_dnp_2}
\end{figure}

\begin{figure}[h]
   \centering
   \includegraphics[width=0.8\textwidth]{Yamamoto_Steady_2005_Fig_8}
   \caption{Figure 8 from \cite{yamamoto_steady_2005}. Values were produced for
    a graphite moderated MSR of core length 4 m and radius 2 m with 0.7 m long 
    plenums at either end and a 0.6 m thick axial reflector using a
    multi-group diffusion approach.}
   \label{fig:yamamoto_transit_time}
\end{figure}

\subsection{$\beta_{eff}$} \label{ssec:beta}
Just as DNP concentrations in-core are reduced by fuel flow so too is their
contribution to the reactivity of multiplying systems, $\beta_{eff}$.
Table \ref{tbl:mattioda_beta_reduction} shows, just as for DNP concentrations
in-core, that $\beta_{eff}$ is reduced both by a higher fluid flow velocity and
by a longer out of core residence time.

\begin{table}[h]
    \caption{Value of $\beta_{eff}^{flow}/\beta_{eff}^{static}$ given out of
        core residence time, $\tau_{l}$, in seconds, 
        and fluid flow velocity, $v_{fluid}$ in [cm/s]. Data taken from
        the last two rows of table 5
        in \cite{mattioda_effective_2000}. A 1D reactor model with height of 3 m
        modeled while a 3 group neutron diffusion approach is used to generate
        the given data.} 
    \label{tbl:mattioda_beta_reduction}
    \begin{center}
        \begin{tabular}{|c|c|c|c|c|}
            \hline
            $\tau_{l}\rightarrow$ & 0 & 5 & 10 & 15 \\
            \hline
            $v_{fluid} = 60$ & 0.843 & 0.539 & 0.468 & 0.440 \\
            \hline
            $v_{fluid} = 100$ & 0.834 & 0.420 & 0.352 & 0.329 \\
            \hline
        \end{tabular}
    \end{center}
\end{table}

However,
this reduction in $\beta_{eff}$ does not act equally upon all DNP groups. In table
\ref{tbl:dulla_flow_regimes_beta}, regardless of flow regime chosen, it is seen
that the reactivity contribution of longer half-life DNP groups is
reduced more than that of shorter half-life DNP groups. In the case of slug
flow in a MOSART like system the longest lived DNP group experiences a 68\%
reduction in reactivity contribution while the shortest lived DNP group
experiences less than a 1\% reduction. These affects are driven by the increased
chance a long lived DNP has of being swept into the primary loop before it can
decay in core whereas a short lived DNP is much more likely to decay in core
before it even has the chance to leave.

\subsection{$k_{eff}$} \label{ssec:keff}
Given that there is a reactivity reduction associated with increased fluid flow
velocity in CFRs, as discussed in section \ref{ssec:beta}, it would seem to follow
that there would be a commensurate reduction in the multiplication factor of
the system. However, in figure \ref{fig:zhang_velocity_keff} it is seen that the
multiplication factor of the system described in \cite{zhang_development_2009-1}
increases with increasing flow velocity in the presence of a fixed inlet temperature and out-of-core
residence time. \cite{zhang_development_2009-1} attributes
this phenomenon to increased cooling of the core, given the higher fluid flow 
velocity, and subsequent doppler reactivity feedback with this effect
having a greater reactivity contribution than the increased loss of DNPs. That
said, looking at figure \ref{fig:zhang_residence_time_keff}, in which inflow
 velocity and inlet temperature are held constant, the expected reduction of
 $k_{eff}$ with increasing out-of-core residence time is seen.


\begin{figure}[h]
   \centering
   \includegraphics[width=0.8\textwidth]{Zhang_Development_2009-1_Fig_14}
   \caption{Figure 14 from \cite{zhang_development_2009-1}.}
   \label{fig:zhang_velocity_keff}
\end{figure}

\begin{figure}[h]
   \centering
   \includegraphics[width=0.8\textwidth]{Zhang_Development_2009-1_Fig_18}
   \caption{Figure 18 from \cite{zhang_development_2009-1}.} 
   \label{fig:zhang_residence_time_keff}
\end{figure}

\section{Models} \label{sec:models}
As seen in section \ref{sec:physics} the neutronic behavior of CFRs is tightly
coupled to the thermohydraulics of the overall system. This presents unique
challenges to traditional methods for reactor kinetics modeling; particularly
point-kinetics approaches which assume a constant value for $\beta_{eff}$. In
response to these challenges three dominant approaches have arisen; 
fluid-flow-velocity field coupled multi-group, time-resolved, neutron diffusion;
new point reactor kinetics schemes; and quasi-statics; generally all coupled, tightly
or loosely, to a thermohydraulics solver. Recently
\cite{laureau_coupled_2015} has proposed a new method based upon a modified
fission matrix approach using Monte Carlo simulations feeding into
computational fluid dynamics (CFD) simulations. Tables \ref{tbl:codes1}
through \ref{tbl:codes5} detail
the various implementations of these approaches throughout literature.
In the following subsections
each approach will be introduced as will two transient simulations; the pump-start
experiment for the MSRE representing a thermal system - this transient having
the added advantage of a wealth of experimental data - and an unprotected loss
of flow (ULOF) transient in the proposed MSFR representing a fast system.
As mentioned in
section \ref{sec:intro} the effects of the fluid flow velocity on the DNPs in each
system have noticeable differences due to the channel type flow found in many
thermal systems and the typically non-structured flow found in many fast
systems.
It should be noted that the use of a particular code as an example of an
approach does not single that code out as being "better" than its peers; codes
were selected as examples based on both the quality and detail of the selected
transient as it was presented in the code's debut paper and a desire for breadth in the number of codes covered. The suitability of any given code to any given task
is a determination that must be made on a case by case basis. Details pertaining
to a code used for a transient example may be found in tables \ref{tbl:codes1}
through \ref{tbl:codes5} as
well as the relevant literature.
\par With regards to the pump start up transient for the MSRE it is worth noting
that the transient occurs at zero power, the fuel is initially molten and
stationary, the fuel's fluid flow velocity is brought to its nominal value in
10 s by the pump start up, and that the removal of control rods was used to
compensate for the increasing loss of DNPs from the core, though unlike the
instantaneous control rod movement assumed in many codes, 
the control rods in the
MSRE were speed limited \cite{krepel_dyn3d-msr_2007}. It is this inserted
reactivity, or amount of removal of the control rods, that the codes simulating
this transient attempt to match.

\subsection{Multi-group diffusion} \label{ssec:mgd}
In terms of fidelity, within the modelling approaches considered for this
work, multi-group diffusion (MGD) approaches are the gold standard in that they
make the fewest assumptions and simplifications out of any of the models.
This does increase their computational cost, however, making these methods
the most computationally expensive of all the approaches considered - at least
those with computational cost data available. However the multi-group
diffusion problem is solved its structure allows, though not all implementations
include all the physics, for the direct incorporation
of fluid flow effects on both the neutron flux and the DNP concentration as
see in equation \ref{eq:mgd}, including advective effects, the third and
fourth terms, on the neutron flux,
 and equation \ref{eq:mgd_dnp}, including advective and diffusive effects on
 the DNP distribution, the left hand side. In equation \ref{eq:mgd_dnp}
 $Sc_{T}$ is the turbulent Schmidt number and $nu_{T}$ is the eddy viscosity. 

\begin{equation}
\label{eq:mgd}
\begin{split}
\frac{1}{v_{g}} \cdot \frac{\partial \phi_{g}}{\partial t} + \nabla \cdot
    (\phi_{g} \overset{\rightharpoonup}u) = & \nabla \cdot D_{g} \nabla \phi_{g}
    + \mathlarger{\sum}_{g \prime \not= g} \Sigma_{s,g \prime \rightarrow g}
    \phi_{g \prime} + (1-\beta)\chi_{p,g}
    \mathlarger{\sum}_{g \prime = 1}^{G}\frac{1}{k_{eff}}
    (\nu \Sigma_{f})_{g \prime}\phi_{g\prime}\\ & + 
    \mathlarger{\sum}_{i = 1}^{I}\chi_{d,g}\lambda_{i}C_{i} 
     - \Sigma_{a,g}\phi_{g} - \mathlarger{\sum}_{g \prime \not= g} 
    \Sigma_{s,g \rightarrow g \prime} \phi_{g} 
\end{split}
\end{equation}

\begin{equation}
\label{eq:mgd_dnp}
\nabla \cdot (C_{i} \overset{\rightharpoonup}u) - 
    \nabla \cdot \frac{\nu_{T}}{Sc_{T}}\nabla C_{i}
    = \beta_{i} \mathlarger{\sum}_{g = 1}^{G} \frac{1}{k_{eff}} 
    (\nu \Sigma_{f})_{g} \phi_{g} - \lambda_{i} C_{i}
\end{equation}

\subsubsection{thermal system} \label{sssec:mgd_ts}
Representing MGD approaches for the evaluation of thermal systems the DYN3D-MSR
code, as presented in \cite{krepel_dyn3d-msr_2007}, is chosen. While DYN3D-MSR is
capable of modelling DNP distributions in 3D the authors in
\cite{krepel_dyn3d-msr_2007} note that, owing to the MSRE's channel guided flow,
the radial DNP distribution is assumed to be constant for their modeling of the
MSRE. Additionally it should be stated that for many of the codes used in the
MOST project benchmark, \cite{delpech_benchmark_2003}, both the original data used
by ORNL and the JEFF2.2 library were used to generate the DNP data.
The results of the DYN3D-MSR model of the MSRE pump start up transient,
along with the experimental values and those obtained using a similar 1D code,
DYN1D-MSR, are presented in figure \ref{fig:krepel_dyn3d_msre_pump_start}. The most
immediate takeaway is that when using the JEFF2.2 libraries, both DYN3D-MSR and
DYN1D-MSR are able to closely match the experimental values though they appear to
miss the reactivity oscillations later in the transient.

\begin{figure}[h]
   \centering
   \includegraphics[width=0.8\textwidth]{Krepel_DYN3D_2007_Fig_9}
   \caption{Figure 9 from \cite{krepel_dyn3d-msr_2007}.} 
   \label{fig:krepel_dyn3d_msre_pump_start}
\end{figure}

\subsubsection{fast system} \label{sssec:mgd_fs}
Representing MGD approaches for the evaluation of fast systems an unnamed code
from Politecnico di Milano (Polimi), as presented in \cite{fiorina_modelling_2014}, is
chosen. It should be noted that in the implementation of the ULOF transient for
the MSFR in \cite{fiorina_modelling_2014} no other flows, other than the primary
loop flow, are lost. Additionally the pumps in the primary loop are assumed to
coast down exponentially with a time constant of 5 seconds. The simulated transient is
seen in figure \ref{fig:fiorina_msfr_ulof} along with the results from another
code, this one from the Technical University of Delft (TUDelft) with an unknown relation to
the code from \cite{kophazi_development_2009} though both do state the use of a
modified version of DALTON as the neutronics solver. 

\begin{figure}[h]
   \centering
   \includegraphics[width=0.8\textwidth]{Fiorina_Modelling_2014_Fig_11}
   \caption{Figure 9 from \cite{fiorina_modelling_2014}.} 
   \label{fig:fiorina_msfr_ulof}
\end{figure}

\subsection{Point reactor kinetics} \label{ssec:prk}
Point reactor kinetics (PRK) is arguably the most well known and most widespread
method for modelling the kinetics of multiplying systems. Traditionally PRK
has used a fixed value for $\beta_{eff}$ and precursor concentration equations
that do not allow for moving precursors, equations \ref{eq:prk} and
\ref{eq:prk_dnp}.

\begin{equation}
    \label{eq:prk}
    \frac{dn(t)}{dt} = \frac{\rho(t) - \beta_{eff}}{\Lambda} n(t) +
     \sum_{i = 1}^{I} \lambda_{i} C_{i}(t)
\end{equation}

\begin{equation}
    \label{eq:prk_dnp}
    \frac{dC_{i}(t)}{dt} = \frac{\beta_{eff}^{i}}{\Lambda} n(t) -
        \lambda_{i} C_{i}(t)
\end{equation}

While these assumptions would not seem to connect well
with the physics of CFRs, their approximation is not far off from the
predictions of more rigorous models. This is well illustrated by figures
\ref{fig:zhang_prk_flow}, \ref{fig:zhang_prk_temp}, and 
\ref{fig:zhang_prk_reac};
the transient depicted is a ULOF transient but in the MOSART reactor with the
flow being reduced to 4\% of its nominal value in 7 seconds from the start
of the transient. In
the aforementioned figures from \cite{zhang_comparison_2009} "SM" refers
to the SIMMER-III code as seen in table \ref{tbl:codes3} from
\cite{rineiski_kinetics_2005}; "MPM" refers to a delayed point reactor kinetics
model, discussed in section \ref{sssec:dpk}, but particularly from
\cite{suzuki_reactivity-initiated-accident_2008}; and "PM0" and "PM1" both
refer to traditional point reactor kinetics models with PM0 using an
uncorrected $\beta_{eff}$ and PM1 using a corrected $\beta_{eff}$ taken from
the first time-step of the SM method - it is unknown if the subsequent
$\beta_{eff}^{i}$s were adjusted commensurately with the overall $\beta_{eff}$.
PM1 represents a sub-class of PRK methods referred to in this paper as
corrected point kinetics (CPK) or those PRK approaches which use a one-time, or
flow dependent, adjusted $\beta_{eff}$ to in some way account for the
displacement of DNPs.
While a discussion of the differences between these methods, and other point
reactor kinetics based methods, will be differed to section \ref{sec:comp}
it is worth noting for the sake of traditional point reactor kinetics that
while PM0 overshot and PM1 undershot the power and temperature peak values, and
while both undershot the final value for the fuel salt temperature, both methods
predicted the same evolution of the transient as the other, more detailed,
methods.

\begin{figure}[h]
   \centering
   \includegraphics[width=0.8\textwidth]{Zhang_Comparison_2009_Fig_1}
   \caption{Figure 1 from \cite{zhang_comparison_2009}.} 
   \label{fig:zhang_prk_flow}
\end{figure}

\begin{figure}[h]
   \centering
   \includegraphics[width=0.8\textwidth]{Zhang_Comparison_2009_Fig_2}
   \caption{Figure 2 from \cite{zhang_comparison_2009}.} 
   \label{fig:zhang_prk_temp}
\end{figure}

\begin{figure}[h]
   \centering
   \includegraphics[width=0.8\textwidth]{Zhang_Comparison_2009_Fig_3}
   \caption{Figure 3 from \cite{zhang_comparison_2009}.} 
   \label{fig:zhang_prk_reac}
\end{figure}

In the following
sections various point reactor kinetics based methods will be elaborated on;
 while differing researchers
tend to implement each given method in a slightly different way, methods are
grouped by similar characteristics to avoid a meaningless proliferation of
approaches.

\subsubsection{decay point reactor kinetics} \label{sssec:dpk}
What are termed as
decay point reactor kinetics (DPK) in this paper are those implementations
which aim to account for the movement of DNPs through source and sink terms
in the DNP concentration equation as seen in equation \ref{eq:dpk_dnp}
where $\tau_{c}$ is the fuel residence time in the core and $\tau_{l}$ is the
fuel residence time in the primary loop.
\textcolor{red}{JEFF, THIS IS A PRETTY COMMON EQUATION AT THIS POINT,
DO I NEED TO CITE ONE OF THE 10+ PAPERS WHICH USES THIS THING?}

\begin{equation}
    \label{eq:dpk_dnp}
    \frac{dC_{i}(t)}{dt} = \frac{\beta_{eff}^{i}}{\Lambda} n(t) -
        \lambda_{i} C_{i}(t) - \frac{C_{i}(t)}{\tau_{c}} +
        \frac{C_{i}(t - \tau_{l}) e^{-\lambda_{i} \tau_{l}}}{\tau_{c}}
\end{equation}

In addition to approximating the loss of DNPs from the core due to fluid
flow effects all DNP implementations presented in this paper also include
a bias reactivity, subtracted from the numerator in the first right-hand-side
term of equation \ref{eq:prk}, or a one-time corrected $\beta_{eff}$ to account
for the reduction of the contribution of DNPs to reactivity.

\par Thermal System: Representing DPK approaches for the evaluation of thermal
systems is a modified version of RELAP5 from \cite{shi_development_2016} in
which both the reactor kinetics and thermohydraulics modules were modified
to better model CFRs. The MSRE pump start transient as evaluated by
\cite{shi_development_2016} is seen in figure \ref{fig:shi_msre_ps}.

\begin{figure}[h]
   \centering
   \includegraphics[width=0.8\textwidth]{Shi_Development_2016_Fig_3}
   \caption{Figure 3 from \cite{shi_development_2016}.} 
   \label{fig:shi_msre_ps}
\end{figure}

\par Fast System: Representing DPK approaches for the evaluation of fast
systems is a DPK implementation from \cite{guerrieri_investigation_2013}
modelling the ULOF transient in the MSFR. The evolution of this transient is
seen in figure \ref{fig:guerrieri_msfr_ulof} in which the primary pumps are
assumed to coast down exponentially with a time constant of 2 seconds to 5\%
 of their nominal flow - as natural circulation can not be modeled in
\cite{guerrieri_investigation_2013}'s set-up. In figure 
\ref{fig:guerrieri_msfr_ulof} 
EQL refers to the equilibrium fuel composition of the
MSFR while the various BOL indicators refer to postulated start-up fuel
compositions for the MSFR. 

\begin{figure}[h]
   \centering
   \includegraphics[width=0.8\textwidth]{Guerrieri_Investigation_2013_Fig_3_18_A}
   \caption{Figure 3.18 from \cite{guerrieri_investigation_2013}.} 
   \label{fig:guerrieri_msfr_ulof}
\end{figure}


\subsubsection{"I" point kinetics} \label{sssec:spk}
"IPK", no meaning for the "I" was given in \cite{merle-lucotte_physical_2015},
are those point reactor kinetics based methods which seek to address the
movement of DNPs through the use of a moving mesh. In
\cite{merle-lucotte_physical_2015} a fixed mesh is used on which parameters
such as reactivity and fission power are calculated while a moving mesh is
used to track DNPs and the temperature of the flowing fluid with the two
meshes exchanging information to update one another. Details of the method and
the governing equations can be found in \cite{merle-lucotte_physical_2015} while
two transients, each changing the extracted power on the secondary side by
50\% - one less and one more - each simulated with the IPK method and with a
standard PRK method using a flow-informed corrected $\beta_{eff}$ are seen
in figure \ref{fig:lucotte_ipk}.

\begin{figure}[h]
   \centering
   \includegraphics[width=0.8\textwidth]{Merle_Lucotte_Physical_2015_Fig_3_Bottom}
   \caption{Figure 3 (Bottom) from \cite{merle-lucotte_physical_2015}.} 
   \label{fig:lucotte_ipk}
\end{figure}

\subsubsection{modified point kinetics} \label{sssec:mpk}
What are termed as
modified point reactor kinetics (MPK) in this paper are those implementations
which use a point reactor kinetics model derived in the presence of fuel motion
and weighted with some weighting function. A complete derivation of such a
model using the adjoint flux and adjoint delayed-neutron-emissivities can be
found in \cite{dulla_models_2005}. Transients employing such a model were found
only for thermal systems and an example of such can be seen in figure 
\ref{fig:dulla_4x_flow} in which the fluid flow velocity is increased by a factor
of four in an MSRE model beginning at nominal power.
In figure \ref{fig:dulla_4x_flow} "standard pk (a)" refers
to a DPK model in which the parameters have been adjoint weighted without fuel
flow, "standard pk (b)" refers to a DPK model in which the parameters have
been adjoint weighted with fuel flow, "reformulated pk" refers to the re-derived
point kinetics model found in \cite{dulla_models_2005}, and "reference" refers
to an MGD solution.

\begin{figure}[h]
   \centering
   \includegraphics[width=0.8\textwidth]{Dulla_Models_2005_Fig_2_17}
   \caption{Figure 2.17 from \cite{dulla_models_2005}.} 
   \label{fig:dulla_4x_flow}
\end{figure}

\subsection{Quasi-statics} \label{ssec:qs}
Quasi-statics is less of a method on its own and more of an attempt to
reach a compromise between MGD and the various implementations of PRK. Whereas
MGD makes many fewer approximations than does PRK and is more able to better
capture shape or flow dependant transients, figure \ref{fig:dulla_4x_flow},
it is much more computationally intensive than PRK. In quasi-static
implementations, of which there are three mentioned in tables \ref{tbl:codes1}
through \ref{tbl:codes5},
computational cost savings are sought while striving to maintain the accuracy
of the solution. This is accomplished by updating the PRK parameters, in
whichever implementation of PRK is used, periodically throughout the simulation
by running a full MGD calculation of the system at the corresponding points
in time. In this manner changes in the flux shape and DNP distribution can
be caught in greater detail and used to update the PRK method which then 
simulates the transient until the next MGD update.
\par When these updates occur is key to the computational cost savings of
quasi-statics. As \cite{dulla_models_2005} points out, transients which induce
large shape perturbations on the flux will likely require many more MGD
updates, or at least in the presence of the perturbation. Without an adaptive
updating scheme the requirement of frequent shape updates through an MGD
calculation can strongly reduce the computational savings offered by
quasi-statics.

\subsubsection{thermal system} \label{sssec:qs_therm}
Representing quasi-static approaches for the evaluation of thermal systems
is the modified SIMMER-III code, as presented in \cite{rineiski_kinetics_2005}.
Worth noting is the PRK scheme, termed separable point kinetic (SPK) in this
paper, used in the SIMMER-III code. The DNPs are split into two types;
static DNPs represented by equation \ref{eq:spk_static} in which N is the
neutron
population amount, $\beta$ is the delayed neutron fraction, F is the fission
source operator and $\Psi(x,t)$ is the neutron flux shape function;
 movable DNPs represented by
equation \ref{eq:spk_move}; and a unifying equation seen in \ref{eq:spk_uni}.
Additional details can be found in \cite{rineiski_kinetics_2005}. The transient
seen in figure \ref{fig:rineiski_msre_ps} is the same MSRE pump start transient
described in section \ref{sec:models}.

\begin{equation}
    \label{eq:spk_uni}
    C(x,t) = C_{s}(x,t) + C_{m}(x,t)
\end{equation}

\begin{equation}
    \label{eq:spk_static}
    \frac{\partial C_{s}(x,t)}{\partial t} = N \beta F \Psi(x,t) -
        \lambda C_{s}(x,t)
\end{equation}
 
\begin{equation}
    \label{eq:spk_move}
    \frac{\partial C_{m}(x,t)}{\partial t} = - \nabla (u(x,t) C_{s}(x,t)) -
        \lambda C_{m}(x,t) - \nabla(u(x,t) C_{m}(x,t))
\end{equation}

\begin{figure}[h]
   \centering
   \includegraphics[width=0.8\textwidth]{Rineiski_Kinetics_2005_Fig_2}
   \caption{Figure 2 from \cite{rineiski_kinetics_2005}.} 
   \label{fig:rineiski_msre_ps}
\end{figure}

\subsubsection{fast system} \label{sssec:qs_fast}
No published examples of a quasi-static scheme employed to analyze a fast
CFR system were found. This fact certainly does not indicate, however, that
quasi-statics are ill suited for fast CFR kinetics analysis.

\subsection{Fission matrix}
Typically fission matrices are thought of as tools to accelerate the fission
source convergence in Monte Carlo codes. However, in 
\cite{laureau_coupled_2015} a method is proposed and demonstrated which uses
fission matrices as the neutronic diver in a transient simulation. While a
replication and explanation of the derivation for this method is beyond the
scope of this work the underlying principle is to pre-compute the necessary
fission matrices for a given transient through a steady state Monte Carlo
simulation. These matrices are then passed into a CFD code which uses them
to update the neutron flux and power distribution in the modeled system. At
this time there exists no published data concerning the computational cost
of such an approach. In figure \ref{fig:laureau_msfr} a 300pcm reactivity
insertion transient in the MSFR can be seen as modeled by this new approach; the
slight bump in power seen at about 3 seconds is due to DNP rich fuel salt
re-circulation from the initial power spike.

\begin{figure}[h]
   \centering
   \includegraphics[width=0.8\textwidth]{Laureau_Coupled_2015_Fig_3}
   \caption{Figure 3 (middle) from \cite{laureau_coupled_2015}.} 
   \label{fig:laureau_msfr}
\end{figure}

\section{Comparisons of the Various Kinetics Methods}
\label{sec:comp}
In terms of suitability for the kinetics analyses of CFRs there should be no
question that time resolved MGD approaches, including appropriate treatment
for the DNPs which usually includes coupling to a thermohydraulics code, will
produce transient simulations with the highest degree of accuracy among the
approaches presented so far; MGD approaches also tend to be had with the highest
degree of computational effort. Quasi-static approaches on the other hand
show comparable simulation capability to MGD approaches, 
figure \ref{fig:rineiski_msre_ps},  while promising, at
least, modest computational cost savings. \cite{dulla_models_2005} showcases
a variety of transients in a critical and sub-critical MSRE configuration with
differing numbers of shape recalculations used in \cite{dulla_models_2005}'s
quasi-static scheme, all the while providing relative computational costs
 as compared to the MGD method
also employed in the same work. In figure \ref{fig:dulla_4x_qs} the same factor
of four increase in flow velocity transient as described in section
\ref{sssec:mpk} is modeled with the quasi-static scheme from
\cite{dulla_models_2005} using differing numbers of shape, or MGD,
recalculations; noted by "k". In table \ref{tbl:dulla_costs}
\cite{dulla_models_2005} reports relative computational costs, as compared to
MGD, for the quasi-static method using different numbers of shape recalculations.


\begin{figure}[h]
   \centering
   \includegraphics[width=0.8\textwidth]{Dulla_Models_2005_Fig_2_18}
   \caption{Figure 2.18 from \cite{dulla_models_2005}.} 
   \label{fig:dulla_4x_qs}
\end{figure}

\begin{table}[h]
    \caption{From table 2.VI in \cite{dulla_models_2005}. Computational cost
        of the quasi-static method with differing numbers of shape
        recalculations relative to the MGD method.}
    \label{tbl:dulla_costs}
    \begin{center}
        \begin{tabular}{|c|c|c|c|c|c|c|c|}
            \hline
            \# shape recalculations & 1 & 2 & 10 & 50 & 100 & 1000 & MGD \\
            \hline
            relative time & 0.001 & 0.024 & 0.172 & 0.458 & 0.822 & 1.102 &
                1.000 \\
            \hline
        \end{tabular}
    \end{center}
\end{table}

It is evident that while quasi-statics can save computational cost with some
degree of lost accuracy, if implemented poorly it can actually cost more than
a full MGD simulation.

\par Considering PRK perhaps the most telling figures are figures
\ref{fig:zhang_prk_flow}, \ref{fig:zhang_prk_temp}, and 
\ref{fig:zhang_prk_reac}. In these figures it is seen how PM0 and PM1,
standard PRK models excepting that PM1 uses a one time corrected $\beta_{eff}$,
compare to DPK and quasi-static models; at least when modelling a ULOF in
the MOSART reactor. PM1, with its decreased $\beta_{eff}$ and lack of correction
for flow effects predicts a faster decrease in both power and flow, a lower
peak averaged fuel salt temperature, and generally lower reactivity
contributions from both DNPs and temperature feedback mechanisms. These
effects are attributable to the lower fraction of delayed neutrons present
in the PM1 model. With a lower fraction from the outset of the transient there
is a lessened positive reactivity introduction in the earlier parts of the
transient from DNPs which would have been built up in higher numbers during
a time of higher reactor power; ultimately leading to a lower power peak.
On the other hand the PM0 model has almost
the opposite behavior which can largely be attributed to its having the
highest fraction of delayed neutrons out of all the models; its $\beta_{eff}$
has no accounting for flow induced effects on DNPs. The larger
$\beta_{eff}$ in the PM0 model leads to increased reactivity introduction
in the beginning of the transient as more DNPs are able to decay.
\par While PM0 and PM1 bound the other two methods an additional takeaway from
these same figures is the closeness of the performance of the DPK and
quasi-static models. For the most part they both predict the same transient
behaviors and outcomes; the peak temperature they each predict is effectively
the same and with DPK splitting the difference in the final averaged fuel salt
temperature between SM and the PM methods roughly in half. Displaying
a similar trend of precision relative to more detailed methods, the DPK method
 used by \cite{guerrieri_investigation_2013} to produce
the transient simulation seen in figure \ref{fig:guerrieri_msfr_ulof}, though
differing pump coast down time constants were used, produces similar transient
behavior as seen in figure \ref{fig:fiorina_msfr_ulof} for which
\cite{fiorina_modelling_2014} used an MGD method to produce.
\par DPK methods may capture well transient evolutions at power and with a loss
of flow but as evidenced in figure \ref{fig:shi_msre_ps} they may fail to
capture well reactivity oscillations in transients occurring at low or zero power
and with increases in flow rate. PRK methods in general struggle to capture
transients with strong spatial effects, including flow variations, notes 
\cite{dulla_models_2005}. Figure \ref{fig:dulla_4x_flow} backs this statement
up well and lends further backing to the struggle many PRK methods have
in capturing flow induced DNP reactivity oscillations. The DNP methods fail
to capture the behavior of the transient entirely missing the reactivity
reduction caused by the increase in flow. Even the MPK method employed by
\cite{dulla_models_2005}, while capturing the overall direction of the
transient, fails to capture the strong oscillations in power caused by
re-circulation of DNP rich or poor fuel salt. The IPK method, on the other hand,
seems to capture DNP induced reactivity oscillations well, even at nominal
power levels. While figure \ref{fig:lucotte_ipk} shows the oscillations the
IPK method is capable of capturing at higher powers, because of scale effects
it misses the more dramatic oscillations that IPK captures at the low power
version of the same transient. These oscillations can be seen in figure
\ref{fig:lucotte_ipk_reac} where the extracted power reduction transient shows
even greater reactivity oscillations than the mirror transient with increased
power extraction which is in line with the general knowledge, echoed in
\cite{zanetti_development_2016}, that DNP re-circulation effects impact
transient behavior more at lower powers.

\begin{figure}[h]
   \centering
   \includegraphics[width=0.8\textwidth]{Merle_Lucotte_Physical_2015_Fig_3_Top}
   \caption{Figure 3 (Top) from \cite{merle-lucotte_physical_2015}.} 
   \label{fig:lucotte_ipk_reac}
\end{figure}

\par Overall what can be said for methods with which to model CFR kinetics
is that more studies are needed to establish the power ranges and transient
types in which the various PRK approaches capture well the reactor behavior.
In the likely event that even the most rigorous PRK approaches fail to capture
specific transient responses of interest the implementation of time adaptive
quasi-static schemes into system codes will be necessary to allow for future
risk informed transient analysis of CFRs in which many hundreds to thousands
of transient calculations will be needed to be run. In the mean time MGD
approaches will continue to be the first method with which any investigation
into CFR kinetics should look to, from which an appropriate quasi-static or
PRK method can be compared to and accepted for use.

\section{Assumptions} \label{sec:asm}
While the choice of a particular model to simulate CFR kinetics with brings
with it a set of benefits and limitations beyond the choice of a modelling
method the selection and use of specific assumptions can certainly have an
impact on the kinetics evaluation as well. In the following subsections the
impact of certain assumptions on transient behavior will be discussed. These
assumptions will be broken down into two main groups, those that affect the
modelling of the DNP distributions and those which affect the thermohydraulics
modeling of a system.

\subsection{DNP related assumptions} \label{ssec:dnp_asm}
One of the most notable assumptions
that the body of work examined in this paper collectively agreed upon was the
avoidance of 1D flow models for tracking DNPs in open core fast systems. Looking
at tables \ref{tbl:codes1} through \ref{tbl:codes5}
 it can be seen that while many evaluations performed
for channel type thermal systems used 1D, axial approximations of the DNP
flow, no such 1D approximation was made for an open core type fast system.
Indeed, even the widely used 2D, rotationally symmetric models used for many
fast systems have been called into question as noted in
\cite{aufiero_development_2014}; this question being what spurred their efforts
to create their full 3D code. Concerning the effects of 1D versus 3D modelling
of DNPs in channel type CFRs the authors of this paper could find no works
which modeled both the DNP distributions in 3D and the thermohydraulics of
the system in question. Overall more studies are needed to investigate the
impact, both in channel type and open cores, of reduced dimensionality
modelling with particular attention to the flux and DNP distribution treatment.
\par In many CFRs, particularly open core fast systems, the fuel salt flow is
turbulent in nature. In these systems DNPs are not only displaced by the
bulk movement of the fluid but may also be affected by turbulent diffusion.
Indeed, figure \ref{fig:cheng_diffusion} shows some perturbations of the DNP
distribution for the fourth group of DNPs when including turbulent diffusion
in the DNP modeling for the MOSART reactor.

\begin{figure}[h]
   \centering
   \includegraphics[width=0.8\textwidth]{Cheng_Development_2014_Fig_19}
   \caption{Figure 19 from \cite{cheng_development_2014}. Left is the fourth
   DNP group concentration in core without any fluid flow effects. Middle
   includes advective flow and right includes both advective flow and
   turbulent diffusion.} 
   \label{fig:cheng_diffusion}
\end{figure}

Turbulent diffusion is typically modeled by inclusion of the second term in
equation \ref{eq:mgd_dnp}. However, as \cite{aufiero_calculating_2014} notes,
there exists very little data on turbulent mass transport in many common
CFR fuel fluids - mostly molten fluoride or chloride salts - thus leaving the
actual value of $Sc_{T}$ quite uncertain. In figure \ref{fig:aufiero_sc}
\cite{aufiero_calculating_2014} does a parametric analysis of the impact of the
turbulent Schmidt number on $\beta_{eff}$ for the MSFR. The takeaway from this
figure being that over 2 orders of magnitude for $Sc_{T}$ $\beta_{eff}$ only
changed by about 20 pcm. Another phenomenon that can affect DNP distributions
is molecular diffusion, however, both \cite{aufiero_calculating_2014} and
\cite{cheng_development_2014} agree that molecular diffusion is so much smaller
than turbulent diffusion and advective flow effects on DNP distributions that it
can be ignored all together.

\begin{figure}[h]
   \centering
   \includegraphics[width=0.8\textwidth]{Aufiero_Calculating_2014_Fig_9}
   \caption{Figure 9 from \cite{aufiero_calculating_2014}.}
   \label{fig:aufiero_sc}
\end{figure}

\subsection{Thermohydraulics related assumptions} \label{th_asm}
Just as there is a choice of assumptions to be made with respect to the
modelling of DNP distributions in CFRs there are choices, more choices, to be
made with respect to the modelling of the thermohydraulics. One of the most
common assumptions made in many of the works examined in this paper was that
of slug flow for the velocity profile in the channel type CFRs. This was a
particularly common assumption for those approaches which did not explicitly
model the fuel salt flow in the core but rather ignored thermohydraulics
all together or employed an energy balance to account for temperature effects
on reactivity. In \cite{dulla_interactions_2007} an exhaustive evaluation
of this assumption, as compared with other flow profiles, was undertaken.
The power evolution through an average velocity change, whose profile is found
in the upper right hand corner, can be seen in figure 
\ref{fig:dulla_flow_regimes}
in which it is seen that the power evolution for this transient is 
influenced by the flow regime assumed. The effect of differing flow regimes
on $\beta_{eff}$ and $\beta_{eff}^{i}$ in the MSRE can be seen in table
\ref{tbl:dulla_flow_regimes_beta}. The flow regimes referred to in figure
\ref{fig:dulla_flow_regimes} and table \ref{tbl:dulla_flow_regimes_beta} are
best 
summarized by figures \ref{fig:dulla_flow_map} and \ref{fig:dulla_flow_adf}.

\begin{figure}[h]
   \centering
   \includegraphics[width=0.8\textwidth]{Dulla_Interactions_2007_Fig_1_A}
   \caption{Figure 1.a from \cite{dulla_interactions_2007}. Circles correspond
    to a "parabolic" flow regime, stars to a "step constant outward decreasing"
    (SOD) flow regime, and squares to a "step constant outward increasing" (SOI)
    flow regime.} 
   \label{fig:dulla_flow_map}
\end{figure}

\begin{figure}[h]
   \centering
   \includegraphics[width=0.8\textwidth]{Dulla_Interactions_2007_Fig_1_B}
   \caption{Figure 1.b from \cite{dulla_interactions_2007}. This figure depicts
   the radial profile of flow velocity for several reactor core heights
   corresponding to an "axially developing field" (ADF) flow regime} 
   \label{fig:dulla_flow_adf}
\end{figure}

\begin{figure}[h]
   \centering
   \includegraphics[width=0.8\textwidth]{Dulla_Interactions_2007_Fig_12}
   \caption{Figure 12 from \cite{dulla_interactions_2007}. Please refer to
   figures \ref{fig:dulla_flow_map} and \ref{fig:dulla_flow_adf} for an
   explanation of the flow regimes seen here. In this figure circles correspond
   to a "slug" (flat) flow regime, asterisks to a "parabolic" flow regime, stars to an
   "ADF" flow regime, squares to an "SOD" flow regime, and triangles to an
   "SOI" flow regime.} 
   \label{fig:dulla_flow_regimes}
\end{figure}

\begin{table}[h]
    \caption{Reproduced from Table III of \cite{dulla_interactions_2007}. "Slug"
        refers to a uniform velocity profile across the reactor core.
        The other flow regimes seen in this table are explained in
        figures \ref{fig:dulla_flow_map} and \ref{fig:dulla_flow_adf}.} 
    \label{tbl:dulla_flow_regimes_beta}
    \begin{center}
        \begin{tabular}{|c|c|c|c|c|c|c|c|c|c|}
            \hline
            \multirow{2}{*}{Flow Regime} & \multirow{2}{*}{$\Delta \rho$(pcm)}
            &\multirow{2}{*}{$\lambda(s^{-1})$}&1 & 2 & 3 & 4 & 5 & 6 &
             \multirow{2}{*}{Total} \\
            \cline{4 - 9}
            & & & 0.0127 & 0.0317 & 0.116 & 0.311 & 1.4 & 3.87 & \\
            \hline
            No flow & 0 & $\beta$ (pcm) & 23.7 & 122.7 & 117.1 & 262.7 & 107.9 &
                45.1 & 679.2 \\
            \hline
            Slug & -251.5 & $\beta_{eff}^{i}$ (pcm) & 7.5 & 41.8 & 52.7 & 178.2&
                 103.5 & 44.9 & 428.6 \\
            & & $\beta_{eff}^{i}/\beta^{i}$ & 0.318 & 0.341 & 0.450 & 0.678 &
                0.959 & 0.995 & 0.631 \\
            \hline
            Parabolic & -299.6 & $\beta_{eff}^{i}$ (pcm) & 7.4 & 39.6 & 44.2 &
                146.5 & 98.8 & 44.5 & 381.0 \\
            & & $\beta_{eff}^{i}/\beta^{i}$ & 0.313 & 0.323 & 0.378 & 0.558 &
                0.916 & 0.986 & 0.561 \\
            \hline
            ADF & -310.4 & $\beta_{eff}^{i}$ (pcm) & 5.8 & 33.5 & 41.9 & 145.8 &
                 98.7 & 44.5 & 370.2 \\
            & & $\beta_{eff}^{i}/\beta^{i}$ & 0.246 & 0.273 & 0.358 & 0.555 &
                0.915 & 0.986 & 0.545 \\
            \hline
            SOD & -281.2 & $\beta_{eff}^{i}$ (pcm) & 8.0 & 40.9 & 47.0 & 157.9 &
                100.8 & 44.7 & 399.3 \\
            & & $\beta_{eff}^{i}/\beta^{i}$ & 0.336 & 0.334 & 0.402 & 0.601 &
                0.934 & 0.990 & 0.588 \\
            \hline
            SOI & -251.5 & $\beta_{eff}^{i}$ (pcm) & 7.5 & 42.9 & 56.6 & 188.9 &
                104.4 & 45.0 & 445.3 \\
            & & $\beta_{eff}^{i}/\beta^{i}$ & 0.315 & 0.350 & 0.483 & 0.719 &
                0.968 & 0.996 & 0.655 \\
            \hline
        \end{tabular}
    \end{center}
\end{table}

\par Less of an assumption and more of a level of detail in the modeling scheme,
but the question remains, what impact does the inclusion of a secondary loop
have on CFR transient modelling? With respect to channel type CFRs, and
particularly the MSRE, this question can be directly answered thanks to the work
of \cite{zanetti_development_2016}. Figures \ref{fig:zanetti_sec_1mw},
\ref{fig:zanetti_sec_5mw}, and \ref{fig:zanetti_sec_8mw} 
depict reactivity insertion
transients in the MSRE at 1 MW, 5 MW, and 8 MW operating power, respectively,
with reactivity insertions of 24 pcm, 19 pcm, and 13 pcm,
respectively. In all three of these transients inclusion of the secondary loop
increases the power peak and decreases the power minimum. It is clear to see
that when modeling CFR transients exclusion of any additional heat transfer
loops is not a conservative assumption and leads to less accurate simulation
results.

\begin{figure}[h]
   \centering
   \includegraphics[width=0.8\textwidth]{Zanetti_Development_2016_Fig_2_5_A}
   \caption{Figure 2.5a from \cite{zanetti_development_2016}.} 
   \label{fig:zanetti_sec_1mw}
\end{figure}

\begin{figure}[h]
   \centering
   \includegraphics[width=0.8\textwidth]{Zanetti_Development_2016_Fig_2_6_A}
   \caption{Figure 2.6a from \cite{zanetti_development_2016}.} 
   \label{fig:zanetti_sec_5mw}
\end{figure}

\begin{figure}[h]
   \centering
   \includegraphics[width=0.8\textwidth]{Zanetti_Development_2016_Fig_2_7_A}
   \caption{Figure 2.5a from \cite{zanetti_development_2016}.} 
   \label{fig:zanetti_sec_8mw}
\end{figure}


\section{Conclusion} \label{sec:conc}
In this paper an overview of the key physical phenomenon affected by fluid fuel
flow in CFRs has been presented, with specific emphasis on the effects seen
in the DNP distribution which is strongly affected by fluid fuel flow.
A review of the current models and codes used to
model CFR kinetics has been provided with sample transients provided in both
fast and thermal systems as produced by representative codes from each major
modelling group. While MGD models are undisputed in their accuracy as compares
to other approaches the suitability of various PRK models was addressed
and they were found to produce results with varying degrees of resemblance to
those produced by MGD approaches. Quasi-static methods were additionally
reviewed and, when implemented so as to play to their strengths, showed promise
as a compromise between the accuracy of the MGD models and the speed of the PRK
models. A novel fission matrix based approach to kinetics modelling in CFRs was
also presented though further study is required before determinations can be
made. Lastly a few shared assumptions and their validity were investigated and
it was found that more work is needed to assess the differences between 2D 
rotationally symmetric models for open core type systems and more robust 3D
approaches. Furthermore the differences between various flow regimes assumed
in channel type reactors were found to have non-negligible impacts on kinetics
analysis and transient modeling indicating that at least loose coupling with
a thermohydraulics code may greatly improve the performance of kinetics models
for CFRs currently employing the slug flow assumption for their velocity field.

\section{Acknowledgements} \label{sec:ack}
The corresponding author would like to state that: ``This material is based
upon work supported under a Department of Energy, Office of Nuclear Energy,
Integrated University Program Graduate Fellowship". The authors would also
like to thank Dr. Aufiero, Dr. Brown, Dr. Fiorina, Dr. Fratoni, 
Dr. K{\u r}epel, 
Dr. Zanetti, and Dr. Zheng for helpful conversations, clarifications, and
materials.  

\section*{References}

\bibliography{Kinetics}

\begin{appendices}

\section{Computational Tools} \label{app:tools}
Concerning tables \ref{tbl:codes1} through \ref{tbl:codes5}, these tables
 summarizes all computational
tools for modelling CFR kinetics which were
found in the literature and dated later than 1990. Tools are listed by the
work they were found in. If tools were mentioned in multiple works the
work which appeared to be most about their development is considered
the primary work with others found in the ``Used In" row. The
``Reactor" row indicates reactor design(s) which were modeled in
the primary work. The ``General Reactor Input" row indicates, to the best
of the author's knowledge, if the computational tool has a generic modelling
capability or is reactor specific; tools which in theory could be used for
any reactor but do not have a general meshing tool are not considered generic.
The ``Precursor Treatment" row indicates what methods were used to
treat the precursor drift; whether that was a variant of the PRK models
or 1D axial or 2D r-z, or 3D full flow modelling of the DNP movement.
The ``Precursor Diffusion" row indicates if turbulent diffusion was included in
the modeling of the DNP distributing.
The ``N.Method" row indicates the computational method used to perform the
kinetics modelling. The ``N.Dimension" row indicates the spatial dimension that
the neutronics method has while the ``Energy Dim." row indicates the energy
discritization used.
The ``Xs(T)" row indicates if any treatment was used to account for temperature
effects on nuclear cross sections, and what treatment if known. 
The ``TH X Eq" rows indicate whether or not an equation governing ``X" was
included in the kinetics modelling.
The ``Turbulence Treatment" row 
indicates which model, if any, was used for modelling turbulent flow in the
thermohydraulics model; this is not related to the ``Precursor Diffusion." row.
The ``Density Treatment" row indicates what treatment, if any, was used for the
density term in the TH equations. The ``Spatial Discritization" and ``Time
Discritization" rows indicate what type of, if any, discritization scheme was
used for each variable. The ``Validation" row indicates any known verification
or validation efforts taken for the given tool. An ``MSRE" entry indicates
that a simulation of an MSRE experiment was compared to the actual experimental
data while a ``Benchmark-MSRE" entry indicates that the respective tool was
evaluated by the European MOST project who's validation efforts are best
summarized by \cite{delpech_benchmark_2003}.

\begin{landscape}
\begin{table}[H]
    \caption{A partial listing of all computational tools for CFR kinetics modelling
        found in the literature dated later than 1990. Please see appendix 
        \ref{app:tools} for an explanation of this table and tables
        \ref{tbl:codes2} through \ref{tbl:codes2} for the continuations of this
        table.}
    \label{tbl:codes1}
    \begin{center}
        \begin{tabular}{|c c c c c c|}
            \hline
            Reference &
                \cite{aufiero_development_2014} &
                \cite{dulla_models_2005} &
                \cite{fiorina_modelling_2014} &
                \cite{guo_simulations_2013} &
                \cite{guerrieri_investigation_2013} \\
                Used In & \cite{aufiero_calculating_2014} &
                    \cite{dulla_dynamics_2008}, \cite{dulla_interactions_2007},
                    \cite{dulla_neutron_2004}, \cite{dulla_quasi-static_2003},
                    \cite{dulla_quasi-static_2008} & &
                    \cite{zhang_development_2009} &
                    \cite{guerrieri_approach_2013},
                    \cite{cammi_dimensional_2012},
                    \cite{guerrieri_multi-physics_2010},
                    \cite{guerrieri_preliminary_2012},
                    \cite{cammi_transfer_2011}\\
                Code Name & None & DYNAMOSS\tablefootnote{While the
                    author cites \cite{dulla_models_2005} as the source of
                    this name is appears nowhere in \cite{dulla_models_2005}.
                    Rather, the code is named in later works.} & None 
                    \tablefootnote{Code detailed is the PoliMi version} &
                    None & None\tablefootnote{In the reference and the other
                    works cited the authors develop several computational
                    tools of differing dimensionality. Not all are reported
                    here but rather the 0D-0D and multi-physics approaches,
                    in that order. Please refer to the cited works for
                    additional details as the authors provide an extensive
                    analysis of all their models.}\\
                Reactor Analyzed & MSFR & MSRE, MSBR & MSFR & MSRE, MOSART &
                    MSRE, MSBR, MSFR\\
                General Reactor Input & No & Uk. & No & Uk. & No\\
                Precursor Treatment & Full Flow & MPK \tablefootnote{
                    Importance weighted parameters} \tablefootnote{Quasi-Static
                    schemes are given two precursor treatments, first their
                    MGD treatment is given, second their PRK treatment is given}
                    & 2D, r-z &
                    DPK & DPK, Full Flow\\
                Precursor Diffusion & Yes & No & Yes & No & No, Yes\\
                Precursor Groups & 8 & 6 & 8 & 6 & 6, 8\\
                N. Method & MGD & Quasi-Statics & MGD & PRK & PRK, MGD\\
                N. Dimension & 3D & 2D, r-z & 2D, r-z & 0D & 0D, 3D\\
                N. Energy Dim. & 6G & MG & 6G & 0 & 0, MG\\
                Xs(T) & Log & None\tablefootnote{To the best of
                    the author's knowledge} & Log & Uk. & Various\\
                TH Mass Eq & Yes & No & Yes & No & No, Yes\\
                TH Energy Eq & Yes & No & Yes & Yes & Yes\\
                TH Momentum Eq & Yes & No & Yes & No & No, Yes\\
                Turbulence Treatment & RANS, realizable k - $\epsilon$ & N/A &
                   RANS, k-$\epsilon$ & N/A & N/A; RANS, k-$\epsilon$\\
                Density Treatment & Boussinesq & N/A & Compressible & N/A &
                   N/A, Compressible\\
                Spatial Discritization & FVM & FVM & Uk. & N/A & N/A, FVM\\
                Time Discritization & ESDIRK & IDE & Uk. & Uk. & Uk.\\
                Validation & Verified\tablefootnote{Against the codes found in
                    \cite{fiorina_modelling_2014}} & Benchmark-MSRE & No &
                    MSRE & MSRE\\
            \hline
        \end{tabular}
    \end{center}
\end{table}
\end{landscape}

\begin{landscape}
\begin{table}[H]
    \caption{A continued listing of all computational tools for CFR kinetics modelling
        found in the literature dated later than 1990. Please see appendix
        \ref{app:tools} for an explanation of this table as well as table
        \ref{tbl:codes1} and tables \ref{tbl:codes3} through \ref{tbl:codes5}
        for the balance of the computational tools data.}
    \label{tbl:codes2}
    \begin{center}
        \begin{tabular}{|c c c c c c|}
            \hline
            Reference &
                \cite{kophazi_development_2009} &
                \cite{krepel_dyn1d-msr_2005} &
                \cite{krepel_dyn3d-msr_2007} &
                \cite{lapenta_point_2001} &
                \cite{laureau_coupled_2015} \\
                Used In & & \cite{krepel_development_2004} & 
                    \cite{krepel_dynamics_2008} & &\\
                Code Name & DT-MSR & DYN1D-MSR & DYN3D-MSR &
                    None\tablefootnote{The author compares an MPK and MGD
                    approach. Both are presented in that order} & None \\
                Reactor Analyzed & MSRE & MSRE, MSBR & MSRE & Slab & MSFR \\
                General Reactor Input & Yes & Yes & Yes & Yes & No \\
                Precursor Treatment & Axial Flow & Axial Flow &
                    Full Flow\textsuperscript{6} & MPK\textsuperscript{4}, MGD &
                    Full Flow\\
                Precursor Diffusion & No & No & No & No, No & Uk. \\
                Precursor Groups & 6 & Arbitrary & Arbitrary & 6, 6 & N/A \\
                N. Method & MGD & MGD & MGD & PRK, MGD &
                    Transient Fission Matrix \\
                N. Dimension & 3D & 1D & 3D & 0D, 1D & 3D \\
                N. Energy Dim. & 8G & 2G & 2G & 0D, 3G & CE \\
                Xs(T) & Interpolation & Interpolation & Uk. & Uk. & OTFDB \\
                TH Mass Eq & Yes & Yes & Yes & No & Yes \\
                TH Energy Eq & Yes & Yes & Yes & No & Yes \\
                TH Momentum Eq & Yes & No & Yes & No & Yes \\
                Turbulence Treatment & RANS, k-$\epsilon$ & N/A & Uk. & N/A&
                    RANS, k-$\epsilon$\\
                Density Treatment & Boussinesq & Uk. & Uk. & N/A & Boussinesq
                    \\
                Spatial Discritization & FEM & NEM & NEM & Uk. & Uk.\\
                Time Discritization & IEM & Uk. & Uk. & Uk. & Uk.\\
                Validation & Benchmark-MSRE & Benchmark-MSRE & Benchmark-MSRE &
                No & No\\
            \hline
        \end{tabular}
    \end{center}
\end{table}
\end{landscape}

\begin{landscape}
\begin{table}[H]
    \caption{A continued listing of all computational tools for CFR kinetics modelling
        found in the literature dated later than 1990. Please see appendix
        \ref{app:tools} for an explanation of this table as well as tables
        \ref{tbl:codes1}, \ref{tbl:codes2}, \ref{tbl:codes4}, and
        \ref{tbl:codes5}.
        for the balance of the computational tools data.}
    \label{tbl:codes3}
    \begin{center}
        \begin{tabular}{|c c c c c c|}
            \hline
            Reference &
                \cite{lecarpentier_neutronic_2003} &
                \cite{merle-lucotte_physical_2015} &
                \cite{nicolino_coupled_2008} &
                \cite{qiu_coupled_2016} &
                \cite{rineiski_kinetics_2005} \\
                Used In & & & & \cite{zhang_couple_2014} &
                    \cite{wang_development_2003}, \cite{rineiski_kinetics_2005},
                    \cite{wang_molten_2006},
                    \cite{rineiski_safety-related_2006} \\
                Code Name & Cinsf1D & None\tablefootnote{The authors provide
                    two PRK based approaches. Both are detailed here, one
                    after the other} & None & COUPLE \tablefootnote{COUPLE
                    has both an MGD solver and a PRK solver, both methods
                    are detailed here} & SIMMER-III \\
                Reactor Analyzed & AMSTER & MSFR & MOSART & MOSART & MSRE\\
                General Reactor Input & Uk. & Uk. & No & Yes & Yes \\
                Precursor Treatment & Axial Flow & CPK, IPK & Full Flow &
                   Full Flow, MPK\textsuperscript{4} & Full Flow,
                   MPK\textsuperscript{4} \\
                Precursor Diffusion & No & No, No & No & No & No \\
                Precursor Groups & 6 & 7, 7 &  Uk. & Arbitrary & 1 \\
                N. Method & MGD & PRK, PRK & MGD & MGD, PRK & Quasi-Statics \\
                N. Dimension & 1D & 0D, 3D & 2D, r-z & 3D & 3D\\
                N. Energy Dim. & 2G & 0, 0 & Uk. & MG, 0 & MG \\
                Xs(T) & Interpolation & Uk. & Uk. & Uk. & Uk.\\
                TH Mass Eq & No & Uk. & Yes & Yes & Yes\\
                TH Energy Eq & Yes & Yes & Yes & Yes & Yes\\
                TH Momentum Eq & No & Uk. & Yes & Yes & Yes\\
                Turbulence Treatment & N/A & Uk. & Stream Function/Vorticity &
                    RANKS, k-$\epsilon$ & Uk.\\
                Density Treatment & N/A & Uk. & Boussinesq & Uk. & Uk.\\
                Spatial Discritization & FDM & Uk. & FVM & FVM
                    \textsuperscript{5} & Uk.\\
                Time Discritization & Uk. & Uk. & IEM & FID & Uk.\\
                Validation & Benchmark-MSRE & No & No & No & Benchmark-MSRE\\
            \hline
        \end{tabular}
    \end{center}
\end{table}
\end{landscape}

\begin{landscape}
\begin{table}[H]
    \caption{A continued listing of all computational tools for CFR kinetics modelling
        found in the literature dated later than 1990. Please see appendix 
        \ref{app:tools} for an explanation of this table as well as tables
        \ref{tbl:codes1}, \ref{tbl:codes2}, \ref{tbl:codes3}, and
        \ref{tbl:codes5} for the balance of the computational tools data.}
    \label{tbl:codes4}
    \begin{center}
        \begin{tabular}{|c c c c c|}
            \hline
            Reference &
                \cite{shi_development_2016} &
                \cite{suzuki_reactivity-initiated-accident_2008} &
                \cite{wu_coupled_2016} &
                \cite{yamamoto_transient_2006} \\
                Used In & & & & \\
                Code Name & RELAP5 & None & VIOLET\_MSR & None \\
                Reactor Analyzed & MSRE, MSBR, TMSR-LF & FUJI-12 & MSRE &
                    Cylinder \\
                General Reactor Input & Yes & No & Yes & No \\
                Precursor Treatment & DPK & DPK & Full Flow, Uk. & Full Flow \\
                Precursor Diffusion & No & No & No & No \\
                Precursor Groups & Arbitrary & 6 & 6 & 6 \\
                N. Method & PRK & PRK & Quasi-Statics & MGD \\
                N. Dimension & 0D & 2D, r-z & 3D & 2D, r-z \\
                N. Energy Dim. & 0 & 0 & Uk. & 2G\\
                Xs(T) & Uk. & Interpolation & Uk. & Uk. \\
                TH Mass Eq & Yes & Uk. & Yes & Yes \\
                TH Energy Eq & Yes & Uk. & Yes & Yes\\
                TH Momentum Eq & Yes & Uk. & Yes & Yes\\
                Turbulence Treatment & None & Uk. & Uk. & Uk.\\
                Density Treatment & Uk. & Uk. & Uk. & Compressible\\
                Spatial Discritization & Uk. & Uk. & VNM & CVM\\
                Time Discritization & Uk. & Uk. & Uk. & Uk.\\
                Validation & MSRE & No & MSRE & No\\
            \hline
        \end{tabular}
    \end{center}
\end{table}
\end{landscape}

\begin{landscape}
\begin{table}[H]
    \caption{A continued listing of all computational tools for CFR kinetics modelling
        found in the literature dated later than 1990. Please see appendix
        \ref{app:tools} for an explanation of this table as well as tables
        \ref{tbl:codes1}, \ref{tbl:codes2}, \ref{tbl:codes3}, and
        \ref{tbl:codes4} for the balance of the computational tools data.}
    \label{tbl:codes5}
    \begin{center}
        \begin{tabular}{|c c c c c|}
            \hline
            Reference &
                \cite{zanetti_development_2016} &
                \cite{zanetti_extension_2015} &
                \cite{zhang_development_2009-1} &
                \cite{zhuang_studies_2015} \\
                Used In & & & \cite{zhang_analysis_2009} &
                    \cite{zheng_development_2014} \\
                Code Name & None & TRACE & None & MOREL\\
                Reactor Analyzed & MSRE & MSRE & Cylinder & MSRE, TMSR-LF\\
                General Reactor Input & No & Yes & Yes & Yes \\
                Precursor Treatment & Full Flow & Full Flow & Full Flow &
                    Full Flow \\
                Precursor Diffusion & No & No & No & No \\
                Precursor Groups & 1 & Arbitrary & 6 & UK \\
                N. Method & PRK & PRK & MGD & MGD\\
                N. Dimension & 0D & 0D & 2D, r-z & 3D\\
                N. Energy Dim. & 0D & 0D & 2G & MG\\
                Xs(T) & Uk. & Yes & Polynomial & Interpolation\\
                TH Mass Eq & Yes & Yes & Yes & No\\
                TH Energy Eq & Yes & Yes & Yes & No\\
                TH Momentum Eq & Yes & Yes & Yes & No\\
                Turbulence Treatment & Uk. & Uk. & RANS, k-$\epsilon$ & Uk.\\
                Density Treatment & Boussinesq & Uk. & Compressible & N/A\\
                Spatial Discritization & FEM & Uk. & FVM & Uk.\\
                Time Discritization & Uk. & Uk. & Uk. & IFD\\
                Validation & No & MSRE & TWIGL\tablefootnote{See
                    \cite{zhang_development_2009} for an explanation} &
                    2D TWIGL\tablefootnote{See \cite{zhuang_studies_2015} for
                    an explanation}, 3D LMW\textsuperscript{12}, MSRE\\
            \hline
        \end{tabular}
    \end{center}
\end{table}
\end{landscape}

\end{appendices}

\end{document}
