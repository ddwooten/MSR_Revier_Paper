\documentclass[review]{elsarticle}

\usepackage{lineno,hyperref}
\modulolinenumbers[5]

%For inclusion of figures
\usepackage{graphicx}

%For isotope notation
\usepackage{mhchem}

%For equation formatting
\usepackage{amsmath}

%For sizing summations properly
\usepackage{relsize}

\journal{Progress in Nuclear Energy}

%%%%%%%%%%%%%%%%%%%%%%%
%% Elsevier bibliography styles
%%%%%%%%%%%%%%%%%%%%%%%
%% To change the style, put a % in front of the second line of the current style and
%% remove the % from the second line of the style you would like to use.
%%%%%%%%%%%%%%%%%%%%%%%

%% Numbered
%\bibliographystyle{model1-num-names}

%% Numbered without titles
%\bibliographystyle{model1a-num-names}

%% Harvard
%\bibliographystyle{model2-names.bst}\biboptions{authoryear}

%% Vancouver numbered
%\usepackage{numcompress}\bibliographystyle{model3-num-names}

%% Vancouver name/year
%\usepackage{numcompress}\bibliographystyle{model4-names}\biboptions{authoryear}

%% APA style
%\bibliographystyle{model5-names}\biboptions{authoryear}

%% AMA style
%\usepackage{numcompress}\bibliographystyle{model6-num-names}

%% `Elsevier LaTeX' style
\bibliographystyle{elsarticle-num}
%%%%%%%%%%%%%%%%%%%%%%%

\begin{document}

\begin{frontmatter}

\title{A Review of Molten Salt Reactor Kinetics Models}

%% Group authors per affiliation:
\author[ucb]{Daniel Wooten\corref{cor1}}
\ead{danieldavidwooten@gmail.com}
\address[ucb]{4155 Etcheverry Hall, MC 1730, University of California, Berkeley,
    Berkeley, CA 94720-1730}
\cortext[cor1]{Corresponding Author}

\author[ornl]{Jeffrey Powers}
\address[ornl]{Oak Ridge}

\begin{abstract}
This template helps you to create a properly formatted \LaTeX\ manuscript.
\end{abstract}

\begin{keyword}
\texttt{circulating fluid kinetics precurssor}
\MSC[2010] 00-01\sep  99-00
\end{keyword}

\end{frontmatter}

\linenumbers

\section{Itroduction} \label{sec:intro}
Standard stuff about MSRs.

\section{Physics of Circulating Fuel} \label{sec:physics}
In comparison to solid fuel reactors circulating fuel reactors exhibit unique
physics which affect all aspects of reactor kinetics. In the following
sub-sections the key phenomonon unique to the kinetics of circulating fuel
reactors are elaborated upon.

\subsection{Neutron flux} \label{ssec:flux}
While the speed of neutrons, fast or thermal, is typically many orders of magnitude greater than
the flow velocity of the fluid fuel the neutron flux can still experience
perturbations due to this fluid velocity field. These perturbations are always
due to second order effects; generally related to changes in fluid temperature
and distributions of delayed neutron precursors - an affect which will be
discussed in the next sub-section. In \cite{zhang_development_2009-1} a steady
state neutronics and coupled thermalhydraulics code, which includes convective
fluid motion in the neutron flux, is used to investigate the effects of fluid
flow on various quantities of interest related to circulating fuel reactors. In
figure \ref{fig:zhang_2d_flux} \cite{zhang_development_2009-1} shows that the
distribution of the neutron flux, thermal is shown but the conclusions for the fast flux are the same, is not significantly shifted in the direction
of the fluid flow. However, \cite{zhang_development_2009-1} does show in figure \ref{fig:zhang_axial_velocity_flux}, axially, and in figure \ref{fig:zhang_radial_velocity_flux}, radially, how a greater fluid flow
 velocity, $U_{in}$ in the figures, supresses the magnitude of the fast, $\phi_{1}$, and thermal, $\phi_{2}$, neutron fluxes. This affect is
 attributed to a larger transport of delayed neutron precursors outside of the core \cite{zhang_development_2009-1}. The reactor model
 utliized in \cite{zhang_development_2009-1} consists of an axially symetric
 open core type reactor with an outer graphite reflector. In the analyses which
 produced figures \ref{fig:zhang_axial_velocity_flux} and \ref{fig:zhang_radial_velocity_flux} the
 out-of-core residence time for the fuel was held constant despite a changing
 fluid flow velocity.

\begin{figure}[h]
   \centering
   \includegraphics[width=0.8\textwidth]{Zhang_Development_2009-1_Fig_8_B}
   \caption{Figure 8.b from \cite{zhang_development_2009-1}. Countour line
    values are the neutron flux; no units were provided with these values.} 
   \label{fig:zhang_2d_flux}
\end{figure}

\begin{figure}[h]
   \centering
   \includegraphics[width=0.8\textwidth]{Zhang_Development_2009-1_Fig_15_A}
   \caption{Figure 15.a from \cite{zhang_development_2009-1}. 
               Axial
               values are taken at the center of the core.}
   \label{fig:zhang_axial_velocity_flux}
\end{figure}

\begin{figure}[h]
   \centering
   \includegraphics[width=0.8\textwidth]{Zhang_Development_2009-1_Fig_15_B}
   \caption{Figure 15.b from \cite{zhang_development_2009-1}.
               Radial 
               values are taken at the center of the core.}
   \label{fig:zhang_radial_velocity_flux}
\end{figure}

\subsection{Delayed neutron precursor distribution} \label{ssec:dnpd}
In contrast to solid fuel reactors where delayed neutron precursors (DNP) decay
 very close to the positions in which they were born, DNPs in CFRs are displaced
 from their originating positions by fluid flow effects; whether advective or
 diffusive. Furthermore, considering that the transit time through the entire
 core and flow loop system of many proposed CFRs is comparable to the decay
 constants of, 
 at least, the longer DNP groups a signigicant fraction of these DNPs can be
expected to decay outside of the core region; emitting their neutrons in areas
of low or zero neutronic importance. As such $\beta_{eff}$, the reactivity
 introduced into a multiplying system by the neutrons emitted by DNPs
is reduced. In the \ce{^{235}U} fueled MSRE the reduction in $\beta_{eff}$ at
zero power was 212 pcm \cite{delpech_benchmark_2003}.

\par Not only do fluid flow effects dispalce DNPs from their originating positions,
as shown in figure \ref{fig:dulla_msre_dnp_displacement}, these effects also
reduce the neutronic importance of DNPs as shown in figure
\ref{fig:dulla_msre_dnp_importance}. This importance reduction is due to two
effects; the liklihood to decay out of core and the liklihood that, even if the
DNP decays in-core, the emitted neutron will originate in a region of low
importance. In \cite{dulla_models_2005} a multi-group
diffusion scheme modeling the MSRE was used to produce the results in figures
\ref{fig:dulla_msre_dnp_displacement} and \ref{fig:dulla_msre_dnp_importance}.
The effects on DNP distributions, from \cite{zhang_development_2009-1}, caused
by varrying flow velocity while holding out-of-core residence time constant are
shown, axially, in figures \ref{fig:zhang_axial_velocity_dnp_1} and 
\ref{fig:zhang_axial_velocity_dnp_2} and, radially, in figures
\ref{fig:zhang_radial_velocity_dnp_1} and \ref{fig:zhang_radial_velocity_dnp_2}.
It is seen that the DNP groups with longer half-lives are most affected as the
fluid flow has sufficient time to displace them before they decay. For all
affected groups it is seen that a higher fluid velocity reduces DNP group
concentration in the core; more DNPs being moved out of the core where they
eventually decay. In figure \ref{fig:yamamoto_transit_time}, while holding fluid flow velocity constant, the effect of
out-of-core residence time on DNP group concentration
in core is shown. It is seen that with increasing transit time the concentration
of DNP groups in core decreases as more DNPs decay out of core.   

\begin{figure}[h]
   \centering
   \includegraphics[width=0.8\textwidth]{Dulla_Models_2005_Fig_1_9}
   \caption{Figure 1.9 from \cite{dulla_models_2005}. Dashed lines represent
    solid fuel results while solid lines represent flowing fuel results. Values
    are taken in the middle of the core. $C_{n}$ are DNP group concentrations. No
    units were given. Higher n indicates shorter decay half-life.}
   \label{fig:dulla_msre_dnp_displacement}
\end{figure}

\begin{figure}[h]
   \centering
   \includegraphics[width=0.8\textwidth]{Dulla_Models_2005_Fig_2_4}
   \caption{Figure 2.4 from \cite{dulla_models_2005}. Dashed lines represent
    solid fuel results while solid lines represent flowing fuel results. Values
    are taken in the middle of the core. $C_{n}^{+}$ are DNP group importance values. 
    No units were given. Higher n indicates shorter decay half-life.}
   \label{fig:dulla_msre_dnp_importance}
\end{figure}

\begin{figure}[h]
   \centering
   \includegraphics[width=0.8\textwidth]{Zhang_Development_2009-1_Fig_16_A}
   \caption{Figure 16.a from \cite{zhang_development_2009-1}. 
               Axial
               values are taken at the center of the core. $C_{n}$ are DNP
               group concentrations. No units were given. Higher n inidicates
               shorter decay half-life.}
   \label{fig:zhang_axial_velocity_dnp_1}
\end{figure}

\begin{figure}[h]
   \centering
   \includegraphics[width=0.8\textwidth]{Zhang_Development_2009-1_Fig_16_C}
   \caption{Figure 16.c from \cite{zhang_development_2009-1}. 
               Axial
               values are taken at the center of the core. $C_{n}$ are DNP
               group concentrations. No units were given. Higher n inidicates
               shorter decay half-life.}
   \label{fig:zhang_axial_velocity_dnp_2}
\end{figure}

\begin{figure}[h]
   \centering
   \includegraphics[width=0.8\textwidth]{Zhang_Development_2009-1_Fig_16_B}
   \caption{Figure 16.b from \cite{zhang_development_2009-1}. 
               Radial 
               values are taken at the center of the core. $C_{n}$ are DNP
               group concentrations. No units were given. Higher n inidicates
               shorter decay half-life.}
   \label{fig:zhang_radial_velocity_dnp_1}
\end{figure}

\begin{figure}[h]
   \centering
   \includegraphics[width=0.8\textwidth]{Zhang_Development_2009-1_Fig_16_D}
   \caption{Figure 16.d from \cite{zhang_development_2009-1}. 
               Radial
               values are taken at the center of the core. $C_{n}$ are DNP
               group concentrations. No units were given. Higher n inidicates
               shorter decay half-life.}
   \label{fig:zhang_radial_velocity_dnp_2}
\end{figure}

\begin{figure}[h]
   \centering
   \includegraphics[width=0.8\textwidth]{Yamamoto_Steady_2005_Fig_8}
   \caption{Figure 8 from \cite{yamamoto_steady_2005}. Values were produced for
    a graphite moderated MSR of core length 4 m and radius 2 m with 0.7 m long 
    plenums at either end and a 0.6 m thick axial reflector using a
    multigroup diffusion approach.}
   \label{fig:yamamoto_transit_time}
\end{figure}

\subsection{$\beta_{eff}$} \label{ssec:beta}
Just as DNP concentrations in-core are reduced by fuel flow so too is their
contribution to the reactivity of multiplying systems, $\beta_{eff}$.
Table \ref{tbl:mattioda_beta_reduction} shows, just as for DNP concentrations
in-core, that $\beta_{eff}$ is reduced both by a higher fluid flow velocity and
by a longer out of core residence time.

\begin{table}[h]
    \caption{Value of $\beta_{eff}^{flow}/\beta_{eff}^{static}$ given out of
        core residence time, $\tau_{l}$, in seconds, 
        and fluid flow velocity, $v_{fluid}$ in [cm/s]. Data taken from
        the last two rows of table 5
        in \cite{mattioda_effective_2000}. A 1D reactor model with height of 3 m
        modeled with a 3 group neutron diffusion approach is used to generate
        the given data.} 
    \label{tbl:mattioda_beta_reduction}
    \begin{center}
        \begin{tabular}{|c|c|c|c|c|}
            \hline
            $\tau_{l}\rightarrow$ & 0 & 5 & 10 & 15 \\
            \hline
            $v_{fluid} = 60$ & 0.843 & 0.539 & 0.468 & 0.440 \\
            \hline
            $v_{fluid} = 100$ & 0.834 & 0.420 & 0.352 & 0.329 \\
            \hline
        \end{tabular}
    \end{center}
\end{table}

However,
this reduction in $\beta_{eff}$ does not act equally upon all DNP groups. In table
\ref{tbl:dulla_flow_regimes_beta}, regardless of flow regime chosen, it is seen
that the reactivity contribution of longer half-life DNP groups is
reduced more than that of shorter half-life DNP groups. In the case of slug
flow in a MOSART like system the longest lived DNP group experiences a 68\%
reduction in reactivity contribution while the shortest lived DNP group
experiences less than a 1\% reduction. These affects are driven by the increased
chance a long lived DNP has of being swept into the primary loop before it can
decay in core whereas a short lived DNP is much more likeley to decay in core
before it even has the chance to leave.

\subsection{$k_{eff}$} \label{ssec:keff}
Given that there is a reactivity reduction associated with increased fluid flow
velocity in CFRs, as discussed in section \ref{ssec:beta}, it would seem to follow
that there would be a commensurate reduction in the multiplication factor of
the system. However, in figure \ref{fig:zhang_velocity_keff} it is seen that the
multiplication factor of the system described in \cite{zhang_development_2009-1}
increases with increasing flow velocity in the presense of a fixed inlet temperature and out-of-core
residence time. \cite{zhang_development_2009-1} attributes
this phenomenon to increased cooling of the core, given the higher fluid flow 
velocity, and subsequent doppler reactivity feedback with this effect
having a greater reactivity contribution than the increased loss of DNPs. That
said, looking at figure \ref{fig:zhang_residence_time_keff}, in which inflow
 velocity and inlet temperature are held constant, the expected reduction of
 $k_{eff}$ with increasing out-of-core residence time is seen.


\begin{figure}[h]
   \centering
   \includegraphics[width=0.8\textwidth]{Zhang_Development_2009-1_Fig_14}
   \caption{Figure 14 from \cite{zhang_development_2009-1}.}
   \label{fig:zhang_velocity_keff}
\end{figure}

\begin{figure}[h]
   \centering
   \includegraphics[width=0.8\textwidth]{Zhang_Development_2009-1_Fig_18}
   \caption{Figure 18 from \cite{zhang_development_2009-1}.} 
   \label{fig:zhang_residence_time_keff}
\end{figure}

\section{Models}
As seen in section \ref{sec:physics} the neutronic behavior of CFRs is tightly
coupled to the thermalhydraulics of the overall system. This presents unique
challenges to traditional methods for reactor kinetics modeling; particuarly
point-kinetics approaches which assume a constant value for $\beta_{eff}$. In
response to these challenges three dominant approaches have arrisen; 
fluid-flow-velocity field coupled multi-group, time-resolved, neutron diffusion;
modified point reactor kinetics; and quasi-statics; generally all coupled, tightly
or loosley, to a thermalhydrualics solver. Recently
\cite{laureau_coupled_2015} has proposed a new method based upon a modified
fission matrix approach using Monte Carlo simulations feeding into
computational fluid dynamics (CFD) simulations. Table \ref{tbl:codes} details
the various implementations of these approaches throughout literature.
In the following subsections
each approach will be introduced as will two transient simulations; the pump-start
experiment for the MSRE representing a thermal system - this transient having
the added advantage of a wealth of experimental data - and an unprotected loss
of flow (ULOF) transient in the proposed MSFR representing a fast system.
As mentioned in
section \ref{sec:intro} the effects of the fluid flow velocity on the DNPs in each
system have noticable differences due to the channel type flow found in many
thermal systems and the typically non-structured flow found in many fast
systems.
It should be noted that the use of a particular code as an example of an
approach does not single that code out as being "better" than its peers; codes
were selected as examples based on both the quality and detail of the selected
transient as it was presented in the code's debut paper and a desire for bredth in the number of codes covered. The suitability of any given code to any given task
is a determination that must be made on a case by case basis. Details pertaining
to a code used for a transient example may be found in table \ref{tbl:codes} as
well as the relevant literature.
\par With regards to the pump start up transient for the MSRE it is worth noting
that the tranisent occurs at zero power, the fuel is initially molten and
stationary, the fuel's fluid flow velocity is brought to its nominal value in
10 s by the pump start up, and that the removal of control rods was used to
compensate for the increasing loss of DNPs from the core, though unlike the
instentaneous control rod movement assumed in many codes, the control rods in the
MSRE were speed limited \cite{krepel_dyn3d-msr_2007}. It is this inserted
reactivity, or amount of removal of the control rods, that the codes simulating
this transient attempt to match.

\subsection{Multi-group diffusion} \label{ssec:mgd}
In terms of fidelity, within the modelling approaches considered for this
work, multi-group diffusion (MGD) approaches are the gold standard in that they
make the fewest assumptions and simplifications out of any of the models.
This does increase their computational cost, however, making these methods
the most computationally expensive of all the approaches considered - at least
those with computational cost data available. However the multi-group
diffusion problem is solved its structure allows, though not all implementations
include all the physics, for the direct incorporation
of fluid flow effects on both the neutron flux and the DNP concentration as
see in equation \ref{eq:mgd}, including advective effects, the third and
fourth terms, on the neutron flux,
 and equation \ref{eq:mgd_dnp}, including advective and diffusive effects on
 the DNP distribution, the left hand side. In equation \ref{eq:mgd_dnp}
 $Sc_{T}$ is the turbulent Schmidt number and $nu_{T}$ is the eddy viscosity. 

\begin{equation}
\label{eq:mgd}
\begin{split}
\frac{1}{v_{g}} \cdot \frac{\partial \phi_{g}}{\partial t} + \nabla \cdot
    (\phi_{g} \overset{\rightharpoonup}u) = & \nabla \cdot D_{g} \nabla \phi_{g}
    + \mathlarger{\sum}_{g \prime \not= g} \Sigma_{s,g \prime \rightarrow g}
    \phi_{g \prime} + (1-\beta)\chi_{p,g}
    \mathlarger{\sum}_{g \prime = 1}^{G}\frac{1}{k_{eff}}
    (\nu \Sigma_{f})_{g \prime}\phi_{g\prime} +
    \mathlarger{\sum}_{i = 1}^{I}\chi_{d,g}\lambda_{i}C_{i} \\
    & - \Sigma_{a,g}\phi_{g} - \mathlarger{\sum}_{g \prime \not= g} 
    \Sigma_{s,g \rightarrow g \prime} \phi_{g} 
\end{split}
\end{equation}

\begin{equation}
\label{eq:mgd_dnp}
\nabla \cdot (C_{i} \overset{\rightharpoonup}u) - 
    \nabla \cdot \frac{\nu_{T}}{Sc_{T}}\nabla C_{i}
    = \beta_{i} \mathlarger{\sum}_{g = 1}^{G} \frac{1}{k_{eff}} 
    (\nu \Sigma_{f})_{g} \phi_{g} - \lambda_{i} C_{i}
\end{equation}

\subsubsection{thermal system} \label{sssec:mgd_ts}
Representing MGD approaches for the evaulation of thermal systems the DYN3D-MSR
code, as presented in \cite{krepel_dyn3d-msr_2007}, is chosen. While DYN3D-MSR is
capable of modelling DNP distributions in 3D the authors in
\cite{krepel_dyn3d-msr_2007} note that, owing to the MSRE's channel guided flow,
the radial DNP distribution is assumed to be constant for their modeling of the
MSRE. Additionally it should be stated that for many of the codes used in the
MOST project benchmark, \cite{delpech_benchmark_2003}, both the original data used
by ORNL and the JEFF2.2 library were used to generate the DNP data.
The results of the DYN3D-MSR model of the MSRE pump start up transient,
along with the experimental values and those obtained using a similar 1D code,
DYN1D-MSR, are presented in figure \ref{fig:krepel_dyn3d_msre_pump_start}. The most
immediate takeaway is that when using the JEFF2.2 libraries, both DYN3D-MSR and
DYN1D-MSR are able to closley match the experimental values though they appear to
miss the reactivity oscilations later in the transient.

\begin{figure}[h]
   \centering
   \includegraphics[width=0.8\textwidth]{Krepel_DYN3D_2007_Fig_9}
   \caption{Figure 9 from \cite{krepel_dyn3d-msr_2007}.} 
   \label{fig:krepel_dyn3d_msre_pump_start}
\end{figure}

\subsubsection{fast system} \label{sssec:mgd_fs}
Representing MGD approaches for the evaluation of fast systems an unamed code
from Politecnico di Milano (Polimi), as presented in \cite{fiorina_modelling_2014}, is
chosen. It should be noted that in the implementation of the ULOF transient for
the MSFR in \cite{fiorina_modelling_2014} no other flows, other than the primary
loop flow, are lost. Additionally the pumps in the primary loop are assumed to
coast down exponentially with a time constant of 5 s. The simulated transient is
seen in figure \ref{fig:fiorina_msfr_ulof} along with the results from another
code, this one from the Technical University of Delft (TUDelft) with an uknown relation to
the code from \cite{kophazi_development_2009} though both do state the use of a
modified version of DALTON as the neutronics solver. 

\begin{figure}[h]
   \centering
   \includegraphics[width=0.8\textwidth]{Fiorina_Modelling_2014_Fig_11}
   \caption{Figure 9 from \cite{fiorina_modelling_2014}.} 
   \label{fig:fiorina_msfr_ulof}
\end{figure}

\subsection{Point reactor kinetics} \label{ssec:prk}
Point reactor kinetics (PRK) is arguably the most well known and most widespread
method for modelling the kinetics of multiplying systems. Traditionally PRK
has used a fixed value for $\beta_{eff}$ and precursor concentration equations
that do not allow for moving prcursors, equations \ref{eq:prk} and
\ref{eq:prk_dnp}.

\begin{equation}
    \label{eq:prk}
    \frac{dn(t)}{dt} = \frac{\rho(t) - \beta_{eff}}{\Lambda} n(t) +
     \sum_{i = 1}^{I} \lambda_{i} C_{i}(t)
\end{equation}

\begin{equation}
    \label{eq:prk_dnp}
    \frac{dC_{i}(t)}{dt} = \frac{\beta_{eff}^{i}}{\Lambda} n(t) -
        \lambda_{i} C_{i}(t)
\end{equation}

While these assumptions would not seem to connect well
with the physics of CFRs, their approximation is not far off from the
predictions of more rigerous models. This is well illustrated by figures
\ref{fig:zhang_prk_flow}, \ref{zhang_prk_temp}, and \ref{zhang_prk_reac};
the transient dipicted is a ULOF transient but in the MOSART reactor with the
flow being reduced to 4\% of its nominal value in 7 seconds from the start
of the transient. In
the aforementioned figures from \cite{zhang_comparison_2009} "SM" refers
to the SIMMER-III code as seen in table \ref{tbl:codes} from
\cite{rineiski_kinetics_2005}; "MPM" refers to a delayed point reactor kinetics
model, discussed in section \ref{sssec:dpk}, but particuarly from
\cite{suzuki_reactivity-initiated-accident_2008}; and "PM0" and "PM1" both
refer to traditional point reactor kinetics models with PM0 using an
uncorrected $\beta_{eff}$ and PM1 using a corrected $\beta_{eff}$ taken from
the first time-step of the SM method - it is unkown if the subseqent
$\beta_{eff}^{i}$s were adjusted commensureately with the overall $\beta_{eff}$.
While a discussion of the differences between these methods, and other point
reactor kinetics based methods, will be differed to section \ref{sssec:prk_dis}
it is worth noting for the sake of traditional point reactor kinetics that
while PM0 overshot and PM1 undershot the power and temperature peak values, and
while both undershot the final value for the fuel salt temperature, both methods
predicted the same evolution of the transient as the other, more detailed,
methods.

\begin{figure}[h]
   \centering
   \includegraphics[width=0.8\textwidth]{Zhang_Comparison_2009_Fig_1}
   \caption{Figure 1 from \cite{zhang_comparison_2009}.} 
   \label{fig:zhang_prk_flow}
\end{figure}

\begin{figure}[h]
   \centering
   \includegraphics[width=0.8\textwidth]{Zhang_Comparison_2009_Fig_2}
   \caption{Figure 2 from \cite{zhang_comparison_2009}.} 
   \label{fig:zhang_prk_temp}
\end{figure}

\begin{figure}[h]
   \centering
   \includegraphics[width=0.8\textwidth]{Zhang_Comparison_2009_Fig_3}
   \caption{Figure 3 from \cite{zhang_comparison_2009}.} 
   \label{fig:zhang_prk_reac}
\end{figure}

In the following
sections various point reactor kinetics based methods will be elaborated on;
 while differing researchers
tend to implement each given method in a slightly different way, methods are
grouped by similar characteristics to avoid a meaningless proliferation of
approaches.

\subsubsection{decay point reactor kinetics} \label{sssec:dpk}
What are termed as
decay point reactor kinetics (DPK) in this paper are those implementations
which aim to account for the movement of DNPs through source and sink terms
in the DNP concentration equation as seen in equation \ref{eq:dpk_dnp}
where $\tau_{c}$ is the fuel residence time in the core and $\tau_{l}$ is the
fuel residence time in the primary loop.
JEFF, THIS IS A PRETTY COMMON EQUATION AT THIS POINT,
DO I NEED TO CITE ONE OF THE 10+ PAPERS WHICH USES THIS THING?

\begin{equation}
    \label{eq:dpk_dnp}
    \frac{dC_{i}(t)}{dt} = \frac{\beta_{eff}^{i}}{\Lambda} n(t) -
        \lamda_{i} C_{i}(t) - \frac{C_{i}(t)}{\tau_{c} +
        \frac{C_{i}(t - \tau_{l} e^{-\lambda_{i} \tau_{l}}}{\tau_{c}}
\end{equation}

In addition to approximating the loss of DNPs from the core due to fluid
flow effects all DNP implementations presented in this paper also include
a bias reactivity, subtracted from the numerator in the first right-hand-side
term of equation \ref{eq:prk}, or a one-time corrected $\beta_{eff}$ to account
for the reduction of the contribution of DNPs to reactivity.

\par Thermal System: Representing DPK approaches for the evaluation of thermal
systems is a modified version of RELAP5 from \cite{shi_development_2016} in
which both the reactor kinetics and thermalhydraulics moduels were modified
to better model CFRs. The MSRE pump start transient as evaluated by
\cite{shi_development_2016} is seen in figure \ref{shi_msre_ps}.

\begin{figure}[h]
   \centering
   \includegraphics[width=0.8\textwidth]{Shi_Development_2016_Fig_3}
   \caption{Figure 3 from \cite{shi_development_2016}.} 
   \label{fig:shi_msre_ps}
\end{figure}

\par Fast System: Representing DPK approaches for the evaluation of fast
systems is a DPK implementation from \cite{zhang_comparison_2009} modelling the
ULOF transient in the MOSART reactor - no DPK approaches were found which
modeled the ULOF transient in the MSFR. The evolution of this transient can be
seen in figures \ref{zhang_prk_flow}, \ref{zhang_prk_temp}, and
\ref{zhang_prk_reac}; all of which refer to the DPK approach as "MPM".

\subsubsection{"I" point kinetics} \label{sssec:spk}
"IPK", no meaning for the "I" was given in \cite{merle-lucotte_physical_2015},
are those point reactor kinetics based methods which seek to address the
movement of DNPs through the use of a moving mesh. In
\cite{merle-lucotte_physical_2015} a fixed mesh is used on which parameters
such as reactivity and fission power are calculated while a moving mesh is
used to track DNPs and the temperature of the flowing fluid with the two
meshes exchanging information to update one another. Details of the method and
the governing equations can be found in \cite{merle-lucotte_physical_2015} while
two transients, each changing the extracted power on the secondary side by
50\% - one less and one more - each simulated with the IPK method and with a
standard PRK method using a flow-informed corrected $\beta_{eff}$ are seen
in figure \ref{fig:lucotte_ipk}.

\begin{figure}[h]
   \centering
   \includegraphics[width=0.8\textwidth]{Merle_Lucotte_Physical_2015_Fig_3}
   \caption{Figure 3 from \cite{merle-lucotte_physical_2015}.} 
   \label{fig:lucotte_ipk}
\end{figure}

\begin{table}[h]
    \caption{Value of $\beta_{eff}^{flow}/\beta_{eff}^{static}$ given out of
        core residence time, $\tau_{l}$, in seconds,
        and fluid flow velocity, $v_{fluid}$ in [cm/s]. Data taken from
        the last two rows of table 5
        in \cite{mattioda_effective_2000}. A 1D reactor model with height of 3 m
        modeled with a 3 group neutron diffusion approach is used to generate
        the given data.} 
    \label{tbl:dulla_flow_regimes_beta}
    \begin{center}
        \begin{tabular}{|c|c|c|c|c|}
            \hline
            $\tau_{l}\rightarrow$ & 0 & 5 & 10 & 15 \\
            \hline
            $v_{fluid} = 60$ & 0.843 & 0.539 & 0.468 & 0.440 \\
            \hline
            $v_{fluid} = 100$ & 0.834 & 0.420 & 0.352 & 0.329 \\
            \hline
        \end{tabular}
    \end{center}
\end{table}

\begin{table}[h]
    \caption{Value of $\beta_{eff}^{flow}/\beta_{eff}^{static}$ given out of
        core residence time, $\tau_{l}$, in seconds,
        and fluid flow velocity, $v_{fluid}$ in [cm/s]. Data taken from
        the last two rows of table 5
        in \cite{mattioda_effective_2000}. A 1D reactor model with height of 3 m
        modeled with a 3 group neutron diffusion approach is used to generate
        the given data.} 
    \label{tbl:codes}
    \begin{center}
        \begin{tabular}{|c|c|c|c|c|}
            \hline
            $\tau_{l}\rightarrow$ & 0 & 5 & 10 & 15 \\
            \hline
            $v_{fluid} = 60$ & 0.843 & 0.539 & 0.468 & 0.440 \\
            \hline
            $v_{fluid} = 100$ & 0.834 & 0.420 & 0.352 & 0.329 \\
            \hline
        \end{tabular}
    \end{center}
\end{table}

\section*{References}

\bibliography{Kinetics}

\end{document}
