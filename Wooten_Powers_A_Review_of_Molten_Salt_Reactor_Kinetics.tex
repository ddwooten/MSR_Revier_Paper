\documentclass[review]{elsarticle}

\usepackage{lineno,hyperref}
\modulolinenumbers[5]

%For inclusion of figures
\usepackage{graphicx}

%For isotope notation
\usepackage{mhchem}

\journal{Progress in Nuclear Energy}

%%%%%%%%%%%%%%%%%%%%%%%
%% Elsevier bibliography styles
%%%%%%%%%%%%%%%%%%%%%%%
%% To change the style, put a % in front of the second line of the current style and
%% remove the % from the second line of the style you would like to use.
%%%%%%%%%%%%%%%%%%%%%%%

%% Numbered
%\bibliographystyle{model1-num-names}

%% Numbered without titles
%\bibliographystyle{model1a-num-names}

%% Harvard
%\bibliographystyle{model2-names.bst}\biboptions{authoryear}

%% Vancouver numbered
%\usepackage{numcompress}\bibliographystyle{model3-num-names}

%% Vancouver name/year
%\usepackage{numcompress}\bibliographystyle{model4-names}\biboptions{authoryear}

%% APA style
%\bibliographystyle{model5-names}\biboptions{authoryear}

%% AMA style
%\usepackage{numcompress}\bibliographystyle{model6-num-names}

%% `Elsevier LaTeX' style
\bibliographystyle{elsarticle-num}
%%%%%%%%%%%%%%%%%%%%%%%

\begin{document}

\begin{frontmatter}

\title{A Review of Molten Salt Reactor Kinetics}

%% Group authors per affiliation:
\author{Daniel Wooten}
\address{4155 Etcheverry Hall, MC 1730, University of California, Berkeley,
    Berkeley, CA 94720-1730}

\author{Jeffrey Powers}
\address{Oak Ridge}

\begin{abstract}
This template helps you to create a properly formatted \LaTeX\ manuscript.
\end{abstract}

\begin{keyword}
\texttt{circulating fluid kinetics precurssor}
\MSC[2010] 00-01\sep  99-00
\end{keyword}

\end{frontmatter}

\linenumbers

\section{Itroduction} \label{introduction}
Standard stuff about MSRs.

\section{Unique Physics of Circulating Fuel} \label{physics}
In comparison to solid fuel reactors circulating fuel reactors exhibit unique
physics which affect all aspects of reactor kinetics. In the following
sub-sections the key phenomonon unique to the kinetics of circulating fuel
reactors are elaborated upon.

\subsection{Neutron Flux} \label{flux}
While the speed of neutrons, fast or thermal, is typically many orders of magnitude greater than
the flow velocity of the fluid fuel the neutron flux can still experience
perturbations due to this fluid velocity field. These perturbations are always
due to second order effects; generally related to changes in fluid temperature
and distributions of delayed neutron precursors - an affect which will be
discussed in the next sub-section. In \cite{zhang_development_2009-1} a steady
state neutronics and coupled thermalhydraulics code, which includes convective
fluid motion in the neutron flux, is used to investigate the effects of fluid
flow on various quantities of interest related to circulating fuel reactors. In
figure \ref{zhang_2d_flux} \cite{zhang_development_2009-1} shows that the
distribution of the neutron flux, thermal is shown but the conclusions for the fast flux are the same, is not significantly shifted in the direction
of the fluid flow. However, \cite{zhang_development_2009-1} does show in figure \ref{zhang_axial_velocity_flux}
, axially, and in figure \ref{zhang_radial_velocity_flux}, radially, how a greater fluid flow
 velocity, $U_{in}$ in the figures, supresses the magnitude of the fast, $\phi_{1}$, and thermal, $\phi_{2}$, neutron fluxes. This affect is
 attributed to a larger transport of delayed neutron precursors outside of the core \cite{zhang_development_2009-1}. The reactor model
 utliized in \cite{zhang_development_2009-1} consists of an axially symetric
 open core type reactor with an outer graphite reflector. In the analyses which
 produced figures \ref{zhang_axial_velocity_flux} and \ref{zhang_radial_velocity_flux} the
 out of core residence time for the fuel was held constant despite a changing
 fluid flow velocity.

\begin{figure}[h]
   \label{zhang_2d_flux}
   \centering
   \includegraphics[width=0.8\textwidth]{Zhang_Development_2009-1_Fig_8_B}
   \caption{Figure 8.b from \cite{zhang_development_2009-1}. Countour line
    values are the neutron flux; no units were provided with these values. 
\end{figure}

\begin{figure}[h]
   \label{zhang_axial_velocity_flux}
   \centering
   \includegraphics[width=0.8\textwidth]{Zhang_Development_2009-1_Fig_15_A}
   \caption{Figure 15.a from \cite{zhang_development_2009-1}. 
               Axial
               values are taken at the center of the core.}
\end{figure}

\begin{figure}[h]
   \label{zhang_radial_velocity_flux}
   \centering
   \includegraphics[width=0.8\textwidth]{Zhang_Development_2009-1_Fig_15_B}
   \caption{Figure 15.b from \cite{zhang_development_2009-1}.
               Radial 
               values are taken at the center of the core.}
\end{figure}

\subsection{Delayed Neutron Precursor Distribution} \label{dnpd}
In contrast to solid fuel reactors where delayed neutron precursors (DNP) decay
 very close to the positions in which they were born, DNPs in CFRs are displaced
 from their originating positions by fluid flow effects; whether advective or
 diffusive. Furthermore, considering that the transit time through the entire
 core and flow loop system of many proposed CFRs is comparable to the decay
 constants of, 
 at least, the longer DNP groups a signigicant fraction of these DNPs can be
expected to decay outside of the core region; emitting their neutrons in areas
of low or zero neutronic importance. As such $\betta_{eff}$, the reactivity
 introduced into a multiplying system by the neutrons emitted by DNPs
is reduced. In the \ce{^{235}U} fueled MSRE the reduction in $\betta_{eff}$ at
zero power was 212 pcm \cite{delpech_benchmark_2003}.
\subsubsection{Fluid Flow Velocity Effects}
Not only do fluid flow effects dispalce DNPs from their originating positions,
as shown in figure \ref{dulla_msre_dnp_displacement}, these effects also
reduce the neutronic importance of DNPs as shown in figure
\ref{dulla_msre_dnp_importance}. This importance reduction is due to two
effects; the liklihood to decay out of core and the liklihood that, even if the
DNP decays in-core, the emitted neutron will originate in a region of low
importance. In \cite{dulla_models_2005} a multi-group
diffusion scheme modeling the MSRE was used to produce the results in figures
\ref{dulla_msre_dnp_displacement} and \ref{dulla_msre_dnp_importance}.
The effects on DNP distributions, from \cite{zhang_development_2009-1}, caused
by varrying flow velocity while holding out-of-core residence time constant are
shown, axially, in figures \ref{zhang_axial_velocity_dnp_1} and 
\ref{zhang_axial_velocity_dnp_2} and, radially, in figures
\ref{zhang_radial_velocity_dnp_1} and \ref{zhang_radial_velocity_dnp_2}.

\begin{figure}[h]
   \label{dulla_msre_dnp_displacement}
   \centering
   \includegraphics[width=0.8\textwidth]{Dulla_Models_2005_Fig_1_9}
   \caption{Figure 1.9 from \cite{Dulla_Models_2005}. Dashed lines represent
    solid fuel results while solid lines represent flowing fuel results. Values
    are taken in the middle of the core. 
\end{figure}

\begin{figure}[h]
   \label{dulla_msre_dnp_importance}
   \centering
   \includegraphics[width=0.8\textwidth]{Dulla_Models_2005_Fig_2_4}
   \caption{Figure 2.4 from \cite{Dulla_Models_2005}. Dashed lines represent
    solid fuel results while solid lines represent flowing fuel results. Values
    are taken in the middle of the core. 
\end{figure}

\begin{figure}[h]
   \label{zhang_axial_velocity_dnp_1}
   \centering
   \includegraphics[width=0.8\textwidth]{Zhang_Development_2009-1_Fig_16_A}
   \caption{Figure 16.a from \cite{zhang_development_2009-1}. 
               Axial
               values are taken at the center of the core.}
\end{figure}

\begin{figure}[h]
   \label{zhang_axial_velocity_dnp_2}
   \centering
   \includegraphics[width=0.8\textwidth]{Zhang_Development_2009-1_Fig_16_C}
   \caption{Figure 16.c from \cite{zhang_development_2009-1}. 
               Axial
               values are taken at the center of the core.}
\end{figure}

\begin{figure}[h]
   \label{zhang_radial_velocity_dnp_1}
   \centering
   \includegraphics[width=0.8\textwidth]{Zhang_Development_2009-1_Fig_16_B}
   \caption{Figure 15.a from \cite{zhang_development_2009-1}. 
               Radial 
               values are taken at the center of the core.}
\end{figure}

\begin{figure}[h]
   \label{zhang_radial_velocity_dnp_2}
   \centering
   \includegraphics[width=0.8\textwidth]{Zhang_Development_2009-1_Fig_16_D}
   \caption{Figure 15.a from \cite{zhang_development_2009-1}. 
               Radial
               values are taken at the center of the core.}
\end{figure}

\section{Bibliography styles}


There are various bibliography styles available. You can select the style of your choice in the preamble of this document. These styles are Elsevier styles based on standard styles like Harvard and Vancouver. Please use Bib\TeX\ to generate your bibliography and include DOIs whenever available.

\section*{References}

\bibliography{Kinetics}

\end{document}
