\documentclass[review]{elsarticle}

\usepackage{lineno,hyperref}
\modulolinenumbers[5]

\usepackage{graphicx}

\journal{Progress in Nuclear Energy}

%%%%%%%%%%%%%%%%%%%%%%%
%% Elsevier bibliography styles
%%%%%%%%%%%%%%%%%%%%%%%
%% To change the style, put a % in front of the second line of the current style and
%% remove the % from the second line of the style you would like to use.
%%%%%%%%%%%%%%%%%%%%%%%

%% Numbered
%\bibliographystyle{model1-num-names}

%% Numbered without titles
%\bibliographystyle{model1a-num-names}

%% Harvard
%\bibliographystyle{model2-names.bst}\biboptions{authoryear}

%% Vancouver numbered
%\usepackage{numcompress}\bibliographystyle{model3-num-names}

%% Vancouver name/year
%\usepackage{numcompress}\bibliographystyle{model4-names}\biboptions{authoryear}

%% APA style
%\bibliographystyle{model5-names}\biboptions{authoryear}

%% AMA style
%\usepackage{numcompress}\bibliographystyle{model6-num-names}

%% `Elsevier LaTeX' style
\bibliographystyle{elsarticle-num}
%%%%%%%%%%%%%%%%%%%%%%%

\begin{document}

\begin{frontmatter}

\title{A Review of Molten Salt Reactor Kinetics}

%% Group authors per affiliation:
\author{Daniel Wooten}
\address{4155 Etcheverry Hall, MC 1730, University of California, Berkeley,
    Berkeley, CA 94720-1730}

\author{Jeffrey Powers}
\address{Oak Ridge}

\begin{abstract}
This template helps you to create a properly formatted \LaTeX\ manuscript.
\end{abstract}

\begin{keyword}
\texttt{circulating fluid kinetics precurssor}
\MSC[2010] 00-01\sep  99-00
\end{keyword}

\end{frontmatter}

\linenumbers

\section{Itroduction} \label{introduction}
Standard stuff about MSRs.

\section{Unique Physics of Circulating Fuel} \lable{physics}
In comparison to solid fuel reactors circulating fuel reactors exhibit unique
physics which affect all aspects of reactor kinetics. In the following
sub-sections the key phenomonon unique to the kinetics of circulating fuel
reactors are elaborated upon.

\subsection{Neutron Flux} \lable{flux}
While the speed of neutrons, fast or thermal, is typically many order of magnitude greater than
the flow velocity of the fluid fuel the neutron flux can still experience
perturbations due to this fluid velocity field. These perturbations are always
due to second order effects; generally related to changes in fluid temperature
and distributions of delayed neutron precursors - an affect which will be
discussed in the next sub-section. In \cite{zhang_development_2009-1} a steady
state neutronics and coupled thermalhydralics code, which includes convective
fluid motion in the neutron flux, is used to show in figure 
\ref{axial_velocity_flux}
, axially, and in figure \ref{radial_velocity_flux} how a greater fluid flow
 velocity supresses the fast and thermal neutron fluxes. This affect is
 attributed to a larger loss of delayed neutron precursors outside of the core.
  The reactor model
 utliized in \cite{zhang_development_2009-1} consists of an axially symetric
 open core type reactor with an outer graphite reflector. In the analyses which
 produced figures \ref{axial_velocity_flux} and \ref{radial_velocity_flux} the
 out of core residence time for the fuel was held constant despite a changing
 fluid flow velocity.




The author names and affiliations could be formatted in two ways:
\begin{enumerate}[(1)]
\item Group the authors per affiliation.
\item Use footnotes to indicate the affiliations.
\end{enumerate}
See the front matter of this document for examples. You are recommended to conform your choice to the journal you are submitting to.

\section{Bibliography styles}

There are various bibliography styles available. You can select the style of your choice in the preamble of this document. These styles are Elsevier styles based on standard styles like Harvard and Vancouver. Please use Bib\TeX\ to generate your bibliography and include DOIs whenever available.

Here are two sample references: \cite{Feynman1963118,Dirac1953888}.

\section*{References}

\bibliography{mybibfile}

\end{document}
