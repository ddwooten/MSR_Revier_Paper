\documentclass[review]{elsarticle}

\usepackage{lineno,hyperref}
\modulolinenumbers[5]

%For inclusion of figures
\usepackage{graphicx}

%For isotope notation
\usepackage{mhchem}

%For equation formatting
\usepackage{amsmath}

\journal{Progress in Nuclear Energy}

%%%%%%%%%%%%%%%%%%%%%%%
%% Elsevier bibliography styles
%%%%%%%%%%%%%%%%%%%%%%%
%% To change the style, put a % in front of the second line of the current style and
%% remove the % from the second line of the style you would like to use.
%%%%%%%%%%%%%%%%%%%%%%%

%% Numbered
%\bibliographystyle{model1-num-names}

%% Numbered without titles
%\bibliographystyle{model1a-num-names}

%% Harvard
%\bibliographystyle{model2-names.bst}\biboptions{authoryear}

%% Vancouver numbered
%\usepackage{numcompress}\bibliographystyle{model3-num-names}

%% Vancouver name/year
%\usepackage{numcompress}\bibliographystyle{model4-names}\biboptions{authoryear}

%% APA style
%\bibliographystyle{model5-names}\biboptions{authoryear}

%% AMA style
%\usepackage{numcompress}\bibliographystyle{model6-num-names}

%% `Elsevier LaTeX' style
\bibliographystyle{elsarticle-num}
%%%%%%%%%%%%%%%%%%%%%%%

\begin{document}

\begin{frontmatter}

\title{A Review of Molten Salt Reactor Kinetics Models}

%% Group authors per affiliation:
\author[ucb]{Daniel Wooten\corref{cor1}}
\ead{danieldavidwooten@gmail.com}
\address[ucb]{4155 Etcheverry Hall, MC 1730, University of California, Berkeley,
    Berkeley, CA 94720-1730}
\cortext[cor1]{Corresponding Author}

\author[ornl]{Jeffrey Powers}
\address[ornl]{Oak Ridge}

\begin{abstract}
This template helps you to create a properly formatted \LaTeX\ manuscript.
\end{abstract}

\begin{keyword}
\texttt{circulating fluid kinetics precurssor}
\MSC[2010] 00-01\sep  99-00
\end{keyword}

\end{frontmatter}

\linenumbers

\section{Itroduction} \label{sec:intro}
Standard stuff about MSRs.

\section{Physics of Circulating Fuel} \label{sec:physics}
In comparison to solid fuel reactors circulating fuel reactors exhibit unique
physics which affect all aspects of reactor kinetics. In the following
sub-sections the key phenomonon unique to the kinetics of circulating fuel
reactors are elaborated upon.

\subsection{Neutron Flux} \label{ssec:flux}
While the speed of neutrons, fast or thermal, is typically many orders of magnitude greater than
the flow velocity of the fluid fuel the neutron flux can still experience
perturbations due to this fluid velocity field. These perturbations are always
due to second order effects; generally related to changes in fluid temperature
and distributions of delayed neutron precursors - an affect which will be
discussed in the next sub-section. In \cite{zhang_development_2009-1} a steady
state neutronics and coupled thermalhydraulics code, which includes convective
fluid motion in the neutron flux, is used to investigate the effects of fluid
flow on various quantities of interest related to circulating fuel reactors. In
figure \ref{fig:zhang_2d_flux} \cite{zhang_development_2009-1} shows that the
distribution of the neutron flux, thermal is shown but the conclusions for the fast flux are the same, is not significantly shifted in the direction
of the fluid flow. However, \cite{zhang_development_2009-1} does show in figure \ref{fig:zhang_axial_velocity_flux}, axially, and in figure \ref{fig:zhang_radial_velocity_flux}, radially, how a greater fluid flow
 velocity, $U_{in}$ in the figures, supresses the magnitude of the fast, $\phi_{1}$, and thermal, $\phi_{2}$, neutron fluxes. This affect is
 attributed to a larger transport of delayed neutron precursors outside of the core \cite{zhang_development_2009-1}. The reactor model
 utliized in \cite{zhang_development_2009-1} consists of an axially symetric
 open core type reactor with an outer graphite reflector. In the analyses which
 produced figures \ref{fig:zhang_axial_velocity_flux} and \ref{fig:zhang_radial_velocity_flux} the
 out-of-core residence time for the fuel was held constant despite a changing
 fluid flow velocity.

\begin{figure}[h]
   \centering
   \includegraphics[width=0.8\textwidth]{Zhang_Development_2009-1_Fig_8_B}
   \caption{Figure 8.b from \cite{zhang_development_2009-1}. Countour line
    values are the neutron flux; no units were provided with these values.} 
   \label{fig:zhang_2d_flux}
\end{figure}

\begin{figure}[h]
   \centering
   \includegraphics[width=0.8\textwidth]{Zhang_Development_2009-1_Fig_15_A}
   \caption{Figure 15.a from \cite{zhang_development_2009-1}. 
               Axial
               values are taken at the center of the core.}
   \label{fig:zhang_axial_velocity_flux}
\end{figure}

\begin{figure}[h]
   \centering
   \includegraphics[width=0.8\textwidth]{Zhang_Development_2009-1_Fig_15_B}
   \caption{Figure 15.b from \cite{zhang_development_2009-1}.
               Radial 
               values are taken at the center of the core.}
   \label{fig:zhang_radial_velocity_flux}
\end{figure}

\subsection{Delayed Neutron Precursor Distribution} \label{ssec:dnpd}
In contrast to solid fuel reactors where delayed neutron precursors (DNP) decay
 very close to the positions in which they were born, DNPs in CFRs are displaced
 from their originating positions by fluid flow effects; whether advective or
 diffusive. Furthermore, considering that the transit time through the entire
 core and flow loop system of many proposed CFRs is comparable to the decay
 constants of, 
 at least, the longer DNP groups a signigicant fraction of these DNPs can be
expected to decay outside of the core region; emitting their neutrons in areas
of low or zero neutronic importance. As such $\beta_{eff}$, the reactivity
 introduced into a multiplying system by the neutrons emitted by DNPs
is reduced. In the \ce{^{235}U} fueled MSRE the reduction in $\beta_{eff}$ at
zero power was 212 pcm \cite{delpech_benchmark_2003}.

\par Not only do fluid flow effects dispalce DNPs from their originating positions,
as shown in figure \ref{fig:dulla_msre_dnp_displacement}, these effects also
reduce the neutronic importance of DNPs as shown in figure
\ref{fig:dulla_msre_dnp_importance}. This importance reduction is due to two
effects; the liklihood to decay out of core and the liklihood that, even if the
DNP decays in-core, the emitted neutron will originate in a region of low
importance. In \cite{dulla_models_2005} a multi-group
diffusion scheme modeling the MSRE was used to produce the results in figures
\ref{fig:dulla_msre_dnp_displacement} and \ref{fig:dulla_msre_dnp_importance}.
The effects on DNP distributions, from \cite{zhang_development_2009-1}, caused
by varrying flow velocity while holding out-of-core residence time constant are
shown, axially, in figures \ref{fig:zhang_axial_velocity_dnp_1} and 
\ref{fig:zhang_axial_velocity_dnp_2} and, radially, in figures
\ref{fig:zhang_radial_velocity_dnp_1} and \ref{fig:zhang_radial_velocity_dnp_2}.
It is seen that the DNP groups with longer half-lives are most affected as the
fluid flow has sufficient time to displace them before they decay. For all
affected groups it is seen that a higher fluid velocity reduces DNP group
concentration in the core; more DNPs being moved out of the core where they
eventually decay. In figure \ref{fig:yamamoto_transit_time}, while holding fluid flow velocity constant, the effect of
out-of-core residence time on DNP group concentration
in core is shown. It is seen that with increasing transit time the concentration
of DNP groups in core decreases as more DNPs decay out of core.   

\begin{figure}[h]
   \centering
   \includegraphics[width=0.8\textwidth]{Dulla_Models_2005_Fig_1_9}
   \caption{Figure 1.9 from \cite{dulla_models_2005}. Dashed lines represent
    solid fuel results while solid lines represent flowing fuel results. Values
    are taken in the middle of the core. $C_{n}$ are DNP group concentrations. No
    units were given. Higher n indicates shorter decay half-life.}
   \label{fig:dulla_msre_dnp_displacement}
\end{figure}

\begin{figure}[h]
   \centering
   \includegraphics[width=0.8\textwidth]{Dulla_Models_2005_Fig_2_4}
   \caption{Figure 2.4 from \cite{dulla_models_2005}. Dashed lines represent
    solid fuel results while solid lines represent flowing fuel results. Values
    are taken in the middle of the core. $C_{n}^{+}$ are DNP group importance values. 
    No units were given. Higher n indicates shorter decay half-life.}
   \label{fig:dulla_msre_dnp_importance}
\end{figure}

\begin{figure}[h]
   \centering
   \includegraphics[width=0.8\textwidth]{Zhang_Development_2009-1_Fig_16_A}
   \caption{Figure 16.a from \cite{zhang_development_2009-1}. 
               Axial
               values are taken at the center of the core. $C_{n}$ are DNP
               group concentrations. No units were given. Higher n inidicates
               shorter decay half-life.}
   \label{fig:zhang_axial_velocity_dnp_1}
\end{figure}

\begin{figure}[h]
   \centering
   \includegraphics[width=0.8\textwidth]{Zhang_Development_2009-1_Fig_16_C}
   \caption{Figure 16.c from \cite{zhang_development_2009-1}. 
               Axial
               values are taken at the center of the core. $C_{n}$ are DNP
               group concentrations. No units were given. Higher n inidicates
               shorter decay half-life.}
   \label{fig:zhang_axial_velocity_dnp_2}
\end{figure}

\begin{figure}[h]
   \centering
   \includegraphics[width=0.8\textwidth]{Zhang_Development_2009-1_Fig_16_B}
   \caption{Figure 16.b from \cite{zhang_development_2009-1}. 
               Radial 
               values are taken at the center of the core. $C_{n}$ are DNP
               group concentrations. No units were given. Higher n inidicates
               shorter decay half-life.}
   \label{fig:zhang_radial_velocity_dnp_1}
\end{figure}

\begin{figure}[h]
   \centering
   \includegraphics[width=0.8\textwidth]{Zhang_Development_2009-1_Fig_16_D}
   \caption{Figure 16.d from \cite{zhang_development_2009-1}. 
               Radial
               values are taken at the center of the core. $C_{n}$ are DNP
               group concentrations. No units were given. Higher n inidicates
               shorter decay half-life.}
   \label{fig:zhang_radial_velocity_dnp_2}
\end{figure}

\begin{figure}[h]
   \centering
   \includegraphics[width=0.8\textwidth]{Yamamoto_Steady_2005_Fig_8}
   \caption{Figure 8 from \cite{yamamoto_steady_2005}. Values were produced for
    a graphite moderated MSR of core length 4 m and radius 2 m with 0.7 m long 
    plenums at either end and a 0.6 m thick axial reflector using a
    multigroup diffusion approach.}
   \label{fig:yamamoto_transit_time}
\end{figure}

\subsection{$\beta_{eff}$} \label{ssec:beta}
Just as DNP concentrations in-core are reduced by fuel flow so too is their
contribution to the reactivity of multiplying systems, $\beta_{eff}$.
Table \ref{tbl:mattioda_beta_reduction} shows, just as for DNP concentrations
in-core, that $\beta_{eff}$ is reduced both by a higher fluid flow velocity and
by a longer out of core residence time.

\begin{table}[h]
    \caption{Value of $\beta_{eff}^{flow}/\beta_{eff}^{static}$ given out of
        core residence time, $\tau_{l}$, in seconds, 
        and fluid flow velocity, $v_{fluid}$ in [cm/s]. Data taken from
        the last two rows of table 5
        in \cite{mattioda_effective_2000}. A 1D reactor model with height of 3 m
        modeled with a 3 group neutron diffusion approach is used to generate
        the given data.} 
    \label{tbl:mattioda_beta_reduction}
    \begin{center}
        \begin{tabular}{|c|c|c|c|c|}
            \hline
            $\tau_{l}\rightarrow$ & 0 & 5 & 10 & 15 \\
            \hline
            $v_{fluid} = 60$ & 0.843 & 0.539 & 0.468 & 0.440 \\
            \hline
            $v_{fluid} = 100$ & 0.834 & 0.420 & 0.352 & 0.329 \\
            \hline
        \end{tabular}
    \end{center}
\end{table}

However,
this reduction in $\beta_{eff}$ does not act equally upon all DNP groups. In table
\ref{tbl:dulla_flow_regimes_beta}, regardless of flow regime chosen, it is seen
that the reactivity contribution of longer half-life DNP groups is
reduced more than that of shorter half-life DNP groups. In the case of slug
flow in a MOSART like system the longest lived DNP group experiences a 68\%
reduction in reactivity contribution while the shortest lived DNP group
experiences less than a 1\% reduction. These affects are driven by the increased
chance a long lived DNP has of being swept into the primary loop before it can
decay in core whereas a short lived DNP is much more likeley to decay in core
before it even has the chance to leave.

\subsection{$k_{eff}$} \label{ssec:keff}
Given that there is a reactivity reduction associated with increased fluid flow
velocity in CFRs, as discussed in section \ref{ssec:beta}, it would seem to follow
that there would be a commensurate reduction in the multiplication factor of
the system. However, in figure \ref{fig:zhang_velocity_keff} it is seen that the
multiplication factor of the system described in \cite{zhang_development_2009-1}
increases with increasing flow velocity in the presense of a fixed inlet temperature and out-of-core
residence time. \cite{zhang_development_2009-1} attributes
this phenomenon to increased cooling of the core, given the higher fluid flow 
velocity, and subsequent doppler reactivity feedback with this effect
having a greater reactivity contribution than the increased loss of DNPs. That
said, looking at figure \ref{fig:zhang_residence_time_keff}, in which inflow
 velocity and inlet temperature are held constant, the expected reduction of
 $k_{eff}$ with increasing out-of-core residence time is seen.


\begin{figure}[h]
   \centering
   \includegraphics[width=0.8\textwidth]{Zhang_Development_2009-1_Fig_14}
   \caption{Figure 14 from \cite{zhang_development_2009-1}.}
   \label{fig:zhang_velocity_keff}
\end{figure}

\begin{figure}[h]
   \centering
   \includegraphics[width=0.8\textwidth]{Zhang_Development_2009-1_Fig_18}
   \caption{Figure 18 from \cite{zhang_development_2009-1}.} 
   \label{fig:zhang_residence_time_keff}
\end{figure}

\section{Models}
As seen in section \ref{sec:physics} the neutronic behavior of CFRs is tightly
coupled to the thermalhydraulics of the overall system. This presents unique
challenges to traditional methods for reactor kinetics modeling; particuarly
point-kinetics approaches which assume a constant value for $\beta_{eff}$. In
response to these challenges three dominant approaches have arrisen; 
fluid-flow-velocity field coupled multi-group, time-resolved, neutron diffusion;
modified point reactor kinetics; and quasi-statics. Recently
\cite{laureau_coupled_2015} has proposed a new method based upon a modified
fission matrix approach using Monte Carlo simulations feeding into
computational fluid dynamics (CFD) simulations. In the following subsections
each approach will be introduced as will two transient simulations; one for
a thermal reactor system and one for a fast reactor system. As mentioned in
section \ref{sec:intro} the effects of the fluid flow velocity on the DNPs in each
system have noticable differences due to the channel type flow found in many
thermal systems and the typically non-structured flow found in many fast
systems.

\subsection{Multi-Group Diffusion} \label{ssec:mgd}
In terms of fidelity, within the modelling approaches considered for this
work, multi-group diffusion (MGD) approaches are the gold standard in that they
make the fewest assumptions and simplifications out of any of the models.
This does increase their computational cost, however, making these methods
the most computationally expensive of all the approaches considered - at least
those with computational cost data available. However the multi-group
diffusion problem is solved its structure allows for the direct incorporation
of fluid flow effects on both the neutron flux and the DNP concentration as
see in equation \ref{eq:mgd}

\begin{table}[h]
    \caption{Value of $\beta_{eff}^{flow}/\beta_{eff}^{static}$ given out of
        core residence time, $\tau_{l}$, in seconds,
        and fluid flow velocity, $v_{fluid}$ in [cm/s]. Data taken from
        the last two rows of table 5
        in \cite{mattioda_effective_2000}. A 1D reactor model with height of 3 m
        modeled with a 3 group neutron diffusion approach is used to generate
        the given data.} 
    \label{tbl:dulla_flow_regimes_beta}
    \begin{center}
        \begin{tabular}{|c|c|c|c|c|}
            \hline
            $\tau_{l}\rightarrow$ & 0 & 5 & 10 & 15 \\
            \hline
            $v_{fluid} = 60$ & 0.843 & 0.539 & 0.468 & 0.440 \\
            \hline
            $v_{fluid} = 100$ & 0.834 & 0.420 & 0.352 & 0.329 \\
            \hline
        \end{tabular}
    \end{center}
\end{table}

\section*{References}

\bibliography{Kinetics}

\end{document}
