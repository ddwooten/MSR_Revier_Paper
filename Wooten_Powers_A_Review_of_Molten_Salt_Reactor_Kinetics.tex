\documentclass[review]{elsarticle}

\usepackage{lineno,hyperref}
\modulolinenumbers[5]

%For inclusion of figures
\usepackage{graphicx}

%For isotope notation
\usepackage{mhchem}

\journal{Progress in Nuclear Energy}

%%%%%%%%%%%%%%%%%%%%%%%
%% Elsevier bibliography styles
%%%%%%%%%%%%%%%%%%%%%%%
%% To change the style, put a % in front of the second line of the current style and
%% remove the % from the second line of the style you would like to use.
%%%%%%%%%%%%%%%%%%%%%%%

%% Numbered
%\bibliographystyle{model1-num-names}

%% Numbered without titles
%\bibliographystyle{model1a-num-names}

%% Harvard
%\bibliographystyle{model2-names.bst}\biboptions{authoryear}

%% Vancouver numbered
%\usepackage{numcompress}\bibliographystyle{model3-num-names}

%% Vancouver name/year
%\usepackage{numcompress}\bibliographystyle{model4-names}\biboptions{authoryear}

%% APA style
%\bibliographystyle{model5-names}\biboptions{authoryear}

%% AMA style
%\usepackage{numcompress}\bibliographystyle{model6-num-names}

%% `Elsevier LaTeX' style
\bibliographystyle{elsarticle-num}
%%%%%%%%%%%%%%%%%%%%%%%

\begin{document}

\begin{frontmatter}

\title{A Review of Molten Salt Reactor Kinetics}

%% Group authors per affiliation:
\author{Daniel Wooten}
\address{4155 Etcheverry Hall, MC 1730, University of California, Berkeley,
    Berkeley, CA 94720-1730}

\author{Jeffrey Powers}
\address{Oak Ridge}

\begin{abstract}
This template helps you to create a properly formatted \LaTeX\ manuscript.
\end{abstract}

\begin{keyword}
\texttt{circulating fluid kinetics precurssor}
\MSC[2010] 00-01\sep  99-00
\end{keyword}

\end{frontmatter}

\linenumbers

\section{Itroduction} \label{introduction}
Standard stuff about MSRs.

\section{Unique Physics of Circulating Fuel} \label{physics}
In comparison to solid fuel reactors circulating fuel reactors exhibit unique
physics which affect all aspects of reactor kinetics. In the following
sub-sections the key phenomonon unique to the kinetics of circulating fuel
reactors are elaborated upon.

\subsection{Neutron Flux} \label{flux}
While the speed of neutrons, fast or thermal, is typically many orders of magnitude greater than
the flow velocity of the fluid fuel the neutron flux can still experience
perturbations due to this fluid velocity field. These perturbations are always
due to second order effects; generally related to changes in fluid temperature
and distributions of delayed neutron precursors - an affect which will be
discussed in the next sub-section. In \cite{zhang_development_2009-1} a steady
state neutronics and coupled thermalhydraulics code, which includes convective
fluid motion in the neutron flux, is used to investigate the effects of fluid
flow on various quantities of interest related to circulating fuel reactors. In
figure \ref{zhang_2d_flux} \cite{zhang_development_2009-1} show that the
distribution of the neutron flux is not significantly shifted in the direction
of the fluid flow. However, \cite{zhang_development_2009-1} does show in figure \ref{axial_velocity_flux}
, axially, and in figure \ref{radial_velocity_flux}, radially, how a greater fluid flow
 velocity supresses the magnitude of the fast, $\phi_{1}$, and thermal, $\phi_{2}$, neutron fluxes. This affect is
 attributed to a larger transport of delayed neutron precursors outside of the core \cite{zhang_development_2009-1}. The reactor model
 utliized in \cite{zhang_development_2009-1} consists of an axially symetric
 open core type reactor with an outer graphite reflector. In the analyses which
 produced figures \ref{axial_velocity_flux} and \ref{radial_velocity_flux} the
 out of core residence time for the fuel was held constant despite a changing
 fluid flow velocity.

\begin{figure}[h]
   \label{zhang_2d_flux}
   \centering
   \includegraphics[width=0.8\textwidth]{Zhang_Development_2009-1_Fig_15_A}
   \caption{Figure 15.a from \cite{zhang_development_2009-1}. 
               Axial
               values are taken at the center of the core.}
\end{figure}

\begin{figure}[h]
   \label{axial_velocity_flux}
   \centering
   \includegraphics[width=0.8\textwidth]{Zhang_Development_2009-1_Fig_15_A}
   \caption{Figure 15.a from \cite{zhang_development_2009-1}. 
               Axial
               values are taken at the center of the core.}
\end{figure}

\begin{figure}[h]
   \label{radial_velocity_flux}
   \centering
   \includegraphics[width=0.8\textwidth]{Zhang_Development_2009-1_Fig_15_A}
   \caption{Figure 15.b from \cite{zhang_development_2009-1}.
               Radial 
               values are taken at the center of the core.}
\end{figure}

\subsection{Delayed Neutron Precursor Distribution} \label{dnpd}
In contrast to solid fuel reactors where delayed neutron precursors (DNP) decay
 very close to the positions in which they were born, DNPs in CFRs are displaced
 from their originating positions by fluid flow effects; whether advective or
 diffusive. Furthermore, considering that the transit time through the entire core and flow loop system of many proposed CFRs is comparable to the decay constants of,
at least, the longer DNP groups a signigicant fraction of these DNPs can be
expected to decay outside of the core region; emitting their neutrons in areas
of low or zero neutronic importance. As such $\betta_{eff}$, the reactivity
 introduced into a multiplying system by the neutrons emitted by DNPs
is reduced. In the MSRE $\betta_{eff}^{static}$ 

\section{Bibliography styles}

There are various bibliography styles available. You can select the style of your choice in the preamble of this document. These styles are Elsevier styles based on standard styles like Harvard and Vancouver. Please use Bib\TeX\ to generate your bibliography and include DOIs whenever available.

\section*{References}

\bibliography{Kinetics}

\end{document}
